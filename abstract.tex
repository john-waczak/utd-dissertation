
\begin{abstract}


  \textcolor{red}{
    \textbf{Proposal:}\newline
    The rapid pace of global change poses a significant and ever present threat
    to human well-being. To facilitate the development of remediation
    technologies and to enable effective mitigation strategies, we must make
    data-driven decisions. However, the limitations posed by the lack of highly
    available, highly resolved data coupled together with the computational
    difficulties posed by direct simulation of physics at scale severely
    constrains our ability to make the low uncertainty predictions needed to
    meaningfully address these challenges in real time. This dissertation
    presents novel machine learning strategies for combining physics knowledge
    with data driven methods in three key case studies. In the first, we
    demonstrate the ability for a coordinated robotic team to estimate the
    concentration of chemicals-of-concern \textit{in real time} by using machine
    learning to map reflectance spectra captured by an autonomous aerial drone
    directly to chemical concentrations with associated uncertainty estimates.
    In the second study, we present a novel technique for using temporal
    variograms to estimate the intrinsic uncertainty of low cost air quality
    sensors directly from their time series. Additionally, we implement two
    physics informed machine learning methods to model these collected time
    series enabling the identification of acute pollution events by modeling
    them as the result of external forcing. Finally, in the third study, we
    present the most comprehensive analysis of indoor air quality to date, which
    includes multi-component observations, a detailed chemical reaction
    mechanism (including ion chemistry), an extensive evaluation of indoor
    photolysis, and full chemical data assimilation (both 4D Var and a full
    Kalman filter) with detailed multi-component error analysis.
  }

  \textcolor{purple}{
    \textbf{Robot Team Supervised:}\newline
    Inland waters pose a unique challenge for water quality monitoring by remote
    sensing techniques due to their complicated spectral features and
    small-scale variability. At the same time, collecting the reference data
    needed to calibrate remote sensing data products is both time consuming and
    expensive. In this study, we present the further development of a robotic
    team composed of an uncrewed surface vessel (USV) providing in situ
    reference measurements and an unmanned aerial vehicle (UAV) equipped with a
    hyperspectral imager. Together, this team is able to address the limitations
    of existing approaches by enabling the simultaneous collection of
    hyperspectral imagery with precisely collocated in situ data. We showcase
    the capabilities of this team using data collected in a northern Texas pond
    across three days in 2020. Machine learning models for 13~variables are
    trained using the dataset of paired in situ measurements and coincident
    reflectance spectra. These models successfully estimate physical variables
    including temperature, conductivity, pH, and turbidity as well as the
    concentrations of blue--green algae, colored dissolved organic matter
    (CDOM), chlorophyll a, crude oil, optical brighteners, and the ions
    $\mathrm{Ca}^{2+}$, $\mathrm{Cl}^{-}$, and $\mathrm{Na}^{+}$. We extend the
    training procedure to utilize conformal prediction to estimate 90\%
    confidence intervals for the output of each trained model. Maps generated by
    applying the models to the collected images reveal small-scale spatial
    variability within the pond. This study highlights the value of combining
    real-time, in situ measurements together with hyperspectral imaging for the
    rapid characterization of water composition.
  }

  \textcolor{brown}{
    \textbf{Robot Team GTM:}\newline
    Unmanned aerial vehicles equipped with hyperspectral imagers have emerged as
    an essential technology for the characterization of inland water bodies. The
    high spectral and spatial resolutions of these systems enable the retrieval
    of a plethora of optically active water quality parameters via band ratio
    algorithms and machine learning methods. However, fitting and validating
    these models requires access to sufficient quantities of in situ reference
    data which are time-consuming and expensive to obtain. In this study, we
    demonstrate how Generative Topographic Mapping (GTM), a probabilistic
    realization of the self-organizing map, can be used to visualize
    high-dimensional hyperspectral imagery and extract spectral signatures
    corresponding to unique endmembers present in the water.  Using data
    collected across a North Texas pond, we first apply  GTM to visualize the
    distribution of captured reflectance spectra, revealing the small-scale
    spatial variability of the water composition. Next, we demonstrate how the
    nodes of the fitted GTM can be interpreted as unique spectral endmembers.
    Using extracted endmembers together with the normalized spectral similarity
    score, we are able to efficiently map the abundance of nearshore algae, as
    well as the evolution of a rhodamine tracer dye used to simulate water
    contamination by a localized source.
  }

  \textcolor{blue}{
    \textbf{Robot Team GSM:}\newline
    In this paper, we introduce a new model for non-linear endmember extraction
    and spectral unmixing of hyperspectral imagery called Generative Simplex
    Mapping (GSM). The model represents endmember mixing using a latent space
    with points sampled within a $(n-1)$-simplex corresponding to the abundance
    of $n$ unique sources. Points in this latent space are non-linearly mapped
    to reflectance spectra via a flexible function combining linear and
    non-linear mixing. Due to the probabilistic formulation of the GSM, spectral
    variability is also estimated by a precision parameter describing the
    distribution of observed spectra. Model parameters are determined using a
    generalized expectation-maximization algorithm. In the event of purely
    linear mixing, non-linear contributions are naturally driven to zero. The
    GSM outperforms three varieties of non-negative matrix factorization for
    both endmember extraction accuracy and abundance estimation on a synthetic
    data set of linearly mixed spectra from the USGS spectral library. In a
    second experiment, the GSM is applied to real hyperspectral imagery captured
    over a pond in North Texas. The model is able to accurately identify
    spectral signatures corresponding to near-shore algae, water, and rhodmaine
    tracer dye introduced into the pond to simulate water contamination by a
    localized source. Abundance maps generated using the GSM accurately track
    evolution of the dye plume as it mixes into the surrounding water.
  }

  \textcolor{olive}{\textbf{Svensen Dissertation:}\newline
    This thesis describes the Generative Topographic Mapping...\newline
    An important, potential application of the GTM is...\newline
    There are two principal limitations of the GTM...\newline
    The framework provides solid ground for extending the GTM to wider
    contexts...\newline
  }



\end{abstract}

