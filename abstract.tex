
\begin{abstract}


Current methods for the evaluation of water quality are limited by sparse
reference measurements and intermittent satellite observations. Meanwhile, air
quality standards are assessed at annual and 24-hour averages which neglect
the impact of short-term spikes on local pollution exposure.
This dissertation develops physics-based machine learning methods to fill
these gaps. To significantly accelerate water quality assessment, we design an
autonomous robotic team combining drone-based hyperspectral imaging with
collocated, in situ data collection by an autonomous boat. Models are trained to
map observed reflectance spectra into 13 physical,
chemical, ionic, and biochemical water quality parameters. These models are then
deployed to map the small-scale spatial variability of water quality across a North
Texas pond. For scenarios in which specific contaminants are not
known in advance, we utilize unsupervised machine learning to
visualize the distribution of water-leaving reflectance spectra and identify spectral
signatures corresponding to unique sources. We extend this approach by
introducing a novel machine learning method called \textit{Generative Simplex
  Mapping} for nonlinear spectral unmixing. Using real data from a rhodamine
tracer dye release, we demonstrate the ability of this model to successfully
identify localized contaminant sources. Finally, we leverage data from a
distributed network of low-cost air quality monitors to construct time series
models for real time particulate matter measurements. The approach extends the
Hankel Alternative View of Koopman framework to identify acute pollution spikes
and enable multi-step forecasts.



\end{abstract}

