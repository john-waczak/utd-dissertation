
\begin{abstract}


\textcolor{red}{UPDATE NEEDED}

% The rapid pace of global change poses significant challenges to human health
% and well-being. However, directly modeling the full range of physical processes
% affecting water and air quality at scales relevant to individuals remains
% computationally infeasible. This dissertation develops physics-based
% machine learning methods to bridge this gap by leveraging water and air
% quality data for actionable insights.


% We developed a robotic team combining drone-based
% hyperspectral imaging with in situ data collection by an autonomous boat. 





% However, directly modeling the comprehensive assortment of physical processes impacting
% water quality and air quality at scales relevant to individuals remains
% computational infeasible. Data-driven methods are therefore well suited to
% address this challenge.


% - This dissertation develops physics-based machine learning models to fill this
% gap by leveraging physical sensing to extract actionable insights.
% - To this end, we developed a robotic team combining drone-based hyperspectral
% imaging with in situ water quality data collection by an autonomous boat
% - Collocated data from this team are used to successfully train models for 13 physical,
% chemical, ionic, and biochemical water quality parameters
% - Applying these trained models to the hyperspectral imagery collected by the
% drone enables rapid mapping of the the distribution each parameter across a
% water bodies
% - However, in many situations, contaminants may not be known in advanced.
% - We address this, we develop unsupervised machine learning approaches to
% identify unique spectral signatures directly from reflectance spectra.
% - A novel method called Generative Simplex Mapping is introduced to perform
% nonlinear spectral unmixing and endmember extraction from reflectance spectra
% - We simulate a localized contaminant source using rhodamine tracer dye
% - The model is able to successfully extract the dye's spectral signature and
% directly estimate its abundance

% Using collocated data from this team, we train machine learning models to map
% reflectance spectra to over 13 physical, chemical, ionic, and biochemical water
% quality parameters. To address situations for which


% However, directly modeling the complex physics behind changes in water and air
% quality 



  %   \textbf{Proposal:}\newline
  %   The rapid pace of global change poses a significant and ever present threat
  %   to human well-being. To facilitate the development of remediation
  %   technologies and to enable effective mitigation strategies, we must make
  %   data-driven decisions. However, the limitations posed by the lack of highly
  %   available, highly resolved data coupled together with the computational
  %   difficulties posed by direct simulation of physics at scale severely
  %   constrains our ability to make the low uncertainty predictions needed to
  %   meaningfully address these challenges in real time. This dissertation
  %   presents novel machine learning strategies for combining physics knowledge
  %   with data driven methods in three key case studies. In the first, we
  %   demonstrate the ability for a coordinated robotic team to estimate the
  %   concentration of chemicals-of-concern \textit{in real time} by using machine
  %   learning to map reflectance spectra captured by an autonomous aerial drone
  %   directly to chemical concentrations with associated uncertainty estimates.
  %   In the second study, we present a novel technique for using temporal
  %   variograms to estimate the intrinsic uncertainty of low cost air quality
  %   sensors directly from their time series. Additionally, we implement two
  %   physics informed machine learning methods to model these collected time
  %   series enabling the identification of acute pollution events by modeling
  %   them as the result of external forcing. Finally, in the third study, we
  %   present the most comprehensive analysis of indoor air quality to date, which
  %   includes multi-component observations, a detailed chemical reaction
  %   mechanism (including ion chemistry), an extensive evaluation of indoor
  %   photolysis, and full chemical data assimilation (both 4D Var and a full
  %   Kalman filter) with detailed multi-component error analysis.
  % }

  % \textcolor{olive}{\textbf{Svensen Dissertation:}\newline
  %   This thesis describes the Generative Topographic Mapping...\newline
  %   An important, potential application of the GTM is...\newline
  %   There are two principal limitations of the GTM...\newline
  %   The framework provides solid ground for extending the GTM to wider
  %   contexts...\newline
  % }



\end{abstract}

