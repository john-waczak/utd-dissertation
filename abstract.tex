
\begin{abstract}


  \textcolor{red}{
    \textbf{Proposal:}\newline
    The rapid pace of global change poses a significant and ever present threat
    to human well-being. To facilitate the development of remediation
    technologies and to enable effective mitigation strategies, we must make
    data-driven decisions. However, the limitations posed by the lack of highly
    available, highly resolved data coupled together with the computational
    difficulties posed by direct simulation of physics at scale severely
    constrains our ability to make the low uncertainty predictions needed to
    meaningfully address these challenges in real time. This dissertation
    presents novel machine learning strategies for combining physics knowledge
    with data driven methods in three key case studies. In the first, we
    demonstrate the ability for a coordinated robotic team to estimate the
    concentration of chemicals-of-concern \textit{in real time} by using machine
    learning to map reflectance spectra captured by an autonomous aerial drone
    directly to chemical concentrations with associated uncertainty estimates.
    In the second study, we present a novel technique for using temporal
    variograms to estimate the intrinsic uncertainty of low cost air quality
    sensors directly from their time series. Additionally, we implement two
    physics informed machine learning methods to model these collected time
    series enabling the identification of acute pollution events by modeling
    them as the result of external forcing. Finally, in the third study, we
    present the most comprehensive analysis of indoor air quality to date, which
    includes multi-component observations, a detailed chemical reaction
    mechanism (including ion chemistry), an extensive evaluation of indoor
    photolysis, and full chemical data assimilation (both 4D Var and a full
    Kalman filter) with detailed multi-component error analysis.
  }

  \textcolor{olive}{\textbf{Svensen Dissertation:}\newline
    This thesis describes the Generative Topographic Mapping...\newline
    An important, potential application of the GTM is...\newline
    There are two principal limitations of the GTM...\newline
    The framework provides solid ground for extending the GTM to wider
    contexts...\newline
  }



\end{abstract}

