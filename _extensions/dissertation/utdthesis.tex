\documentclass[doublespacing]{utdthesis}
% For one-and-a-half spacing, use: \documentclass[halfspacing]{utdthesis}

%%% Load any desired packages in the space below.
%%% Warning: Do not load packages that change the margins, headers, or footers!
%%%
% Optional: If you want to use Times as your font, load it here.  Note that
% although package "times" should work, it may not be the best choice.  Newer
% LaTeX distributions offer "mathptmx" and "newtxtext,newtxmath" as superior
% replacements.  You should find out which is best for your LaTeX.  (If this
% sounds confusing, you probably shouldn't try to change the font to Times.)
%\usepackage{times}
%
% Optional: If your LaTeX has microtype, use it to improve text quality:
\usepackage{microtype}
%
% Recommended: If your dissertation contains math, use the AMS packages:
\usepackage{amsmath,amssymb,amsthm}
%
% Recommended: If your dissertation needs embedded graphics, use graphicx:
\usepackage{graphicx}
%
% Recommended: If your bibliography contains web page URLs, the url package
% improves their appearance (e.g., better line breaking):
\usepackage{url}
%
% Required: To satisfy UTD's formatting requirements for citations, use the
% "natbib" package.  (Use other citation packages at your own risk; not all
% are flexible enough to meet UTD's requirements.)  If you wish to use numeric
% citations, change "authoryear" to "numbers" below.  Use the "chicago" BibTeX
% style, which most closely matches the Turabian formatting required by UTD.
% UTD mandates a blank line between each pair of bibliography entries, so set
% \bibsep as shown below.  Finally, if you are accustomed to using \cite as
% your citation macro, point it to natbib's \citep macro as shown.
\usepackage[authoryear]{natbib}
\bibliographystyle{chicago}
\setlength{\bibsep}{12pt plus 1pt minus 1pt}
\let\cite=\citep
%
% Required: If you have any wide tables or figures that need to be typeset
% in landscape, use the rotating package:
\usepackage{rotating}
%
% Optional: If you use hyperref to auto-generate hyperlinks, always load it
% LAST since it modifies everything else.  In addition, only load hyperref if
% you use pdftex or pdflatex to generate PDFs directly.  Do NOT use it if you
% use plain tex or latex to generate a DVI file.  (If you are generating DVI
% files which you then convert to PDF, you should seriously consider switching
% to pdflatex.  The DVI format loses information because it cannot support
% modern PDF document features.  Using pdflatex to generate PDFs directly
% therefore results in documents of significantly higher quality.)
\usepackage{ifpdf}
\ifpdf
  \usepackage{hyperref}
\fi
%
%%% End of packages.

%%% Define all your personal macros here (if you have any).
%
\providecommand{\hyperref}[2][]{#2}

\newenvironment{exampleclasscode}
 {\parindent=1cm\vskip0pt plus2pt minus0pt\begin{verse}}
 {\end{verse}\vskip0pt plus2pt minus0pt}
%
%%% End of personal macro definitions.


%%% The following definitions MUST come before the document begins.
%
\author{$author$}
\title{$title$}
\thesistype{$degreetype$}  % or "Thesis"
\degreefull{$degreefull$}
\degreeabbr{$degreeabbr$}
\subject{$subject$}
\graduationmonth{$graduationmonth$}
\graduationyear{$graduationyear$}
\prevdegrees{$previousdegrees$} % comma-separated list of PREVIOUS degrees

% List committee members in order.  Mark chairpersons with a "*":
\committeemember*{$committeechair$}
\committeemember{$committee1$}
\committeemember{$committee2$}
\committeemember{$committee3$}
%
%%% End of definitions.


%%% Beginning of actual thesis document.

\begin{document}

\frontmatter

\signaturepage

\copyrightpage{$copyright$} % optional

\begin{dedication} % optional
  $dedication$
\end{dedication}

\maketitle

\begin{acks}{December 2012} % date when thesis first submitted to committee
  $acknowledgements$
\end{acks}

\begin{abstract}
  $abstract$
\end{abstract}

\tableofcontents
\listoffigures % required if you have any figures
\listoftables % required if you have any tables

\mainmatter


$body$

%% \appendix % required only if you have appendixes

%% \chapter*{Sample Solo Appendix}
%% \label{a:other}

%% This appendix illustrates the typesetting of a solo appendix, as specified
%% in \S\ref{s:appendixes}.
%% Solo appendixes are not labeled (although any constituent subsections,
%% tables, or figures are labeled as if the appendix is labeled ``A'').


% Begin the bibliography:

\begin{thesisbib}  % <--- THIS LINE IS REQUIRED!
  \bibliography{$bibliography$}
\end{thesisbib}  % <-- THIS LINE IS REQUIRED!


\begin{biosketch}
  $biosketch$
\end{biosketch}


\begin{vita}  % <-- THIS LINE IS REQUIRED!

  % Replace the lines below with your CV using any formatting you wish,
  % or put nothing in this section and replace these pages with your CV
  % in the resulting PDF file.  (But you MUST include the \begin{vita}
  % and \end{vita} lines even if you intend to replace the pages, since
  % those lines are needed to put the Curriculum Vitae entry into the
  % Table of Contents.)

  \begin{center}
    {\LARGE\bfseries Kevin W.~Hamlen} \\[5pt]
    October 15, 2016
  \end{center}

  \bigskip

  {\large\bfseries Contact Information:\par}
  \medskip
  \noindent\vtop{\hsize=.49\hsize
    Department of Computer Science\par
    The University of Texas at Dallas\par
    800 W.~Campbell Rd.\par
    Richardson, TX 75080-3021, U.S.A.\par}
  \hfil\vtop{\hsize=.49\hsize
    Voice: (972) 883-4724\par
    Fax: (972) 883-2349\par
    Email: \texttt{hamlen@utdallas.edu}\par}\par

  \bigskip

  {\large\bfseries Educational History:\par}
  \medskip
  B.S., Computer Science and Mathematical Sciences,
    Carnegie Mellon University, 1998\par
  M.S., Computer Science, Cornell University, 2002\par
  Ph.D., Computer Science, Cornell University, 2006\par
  \medskip
  \textit{Security Policy Enforcement by Program-rewriting}\par
  Ph.D.~Dissertation\par
  Computer Science Department, Cornell University\par
  Advisors: Dr.~Greg Morrisett and Dr.~Fred B.~Schneider\par
  \medskip
  \textit{Proof-Carrying Code for x86 Architectures}\par
  Senior Undergraduate Honors Thesis\par
  School of Computer Science, Carnegie Mellon University\par
  Advisor: Dr.~Peter Lee

  \bigskip

  {\large\bfseries Employment History:\par}
  \medskip
  Professor, The University of Texas at Dallas,
    September 2018~-- present\par
  Associate Professor, The University of Texas at Dallas,
    September 2012~-- September 2018\par
  Assistant Professor, The University of Texas at Dallas,
    August 2006~-- August 2012\par

  \bigskip

  {\large\bfseries Professional Recognitions and Honors:\par}
  \medskip
  Louis A.~Beecherl, Jr., Faculty Endowment Award, UTD, 2020\par
  Eugene McDermott Faculty Endowment Award, UTD, 2018\par
  Outstanding Teaching Award, Engineering and Computer Science, UTD, 2013\par
  Faculty Research Award, Engineering and Computer Science, UTD, 2012\par
  CAREER Award, National Science Foundation, 2011\par
  Young Investigator Program (YIP) Award, AFOSR, 2008\par
  Allen Newell Award for Excellence in Undergraduate Research,
    Carnegie Mellon U., 1998\par
  Graduated \textit{summa cum laude} (3rd in class), Carnegie Mellon University, 1998\par

  \bigskip

  {\large\bfseries Professional Memberships:\par}
  \medskip
  Institute of Electrical and Electronics Engineers (IEEE), 2010--present\par
  Association of Computing Machinery (ACM), 2008--present\par

\end{vita}  % <-- THIS LINE IS REQUIRED!


\end{document}

