\chapter{Nonlinear Endmember Extraction and Spectral Unmixing with Generative Simplex Mapping}\label{ch:robot-team-gsm}


\textcolor{red}{
  In this paper, we introduce a new model for non-linear endmember extraction
  and spectral unmixing of hyperspectral imagery called Generative Simplex
  Mapping (GSM). The model represents endmember mixing using a latent space
  with points sampled within a $(n-1)$-simplex corresponding to the abundance
  of $n$ unique sources. Points in this latent space are non-linearly mapped
  to reflectance spectra via a flexible function combining linear and
  non-linear mixing. Due to the probabilistic formulation of the GSM, spectral
  variability is also estimated by a precision parameter describing the
  distribution of observed spectra. Model parameters are determined using a
  generalized expectation-maximization algorithm. In the event of purely
  linear mixing, non-linear contributions are naturally driven to zero. The
  GSM outperforms three varieties of non-negative matrix factorization for
  both endmember extraction accuracy and abundance estimation on a synthetic
  data set of linearly mixed spectra from the USGS spectral library. In a
  second experiment, the GSM is applied to real hyperspectral imagery captured
  over a pond in North Texas. The model is able to accurately identify
  spectral signatures corresponding to near-shore algae, water, and rhodmaine
  tracer dye introduced into the pond to simulate water contamination by a
  localized source. Abundance maps generated using the GSM accurately track
  evolution of the dye plume as it mixes into the surrounding water.
}



\section{Motivation}

\section{Spectral Mixing Models}
\subsection{Linear Mixing}
\subsection{Bilinear Mixing}
\subsection{Postlinear Polynomial Mixing}
\subsection{Popular Models}
NCA, PPI, MVF, Autoencoders, etc...

\section{Nonnegative Matrix Factorization}

\section{GSM}

\section{Study Overview}

\section{Results}

\section{Discussion}

