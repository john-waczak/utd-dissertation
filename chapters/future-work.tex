\chapter{Future Work}\label{ch:future-work}

In this chapter, we present directions for future research which can expand and
extend the methods developed in this dissertation.

\section{Robot Team}

The robot team described in Chapters~\ref{ch:robot-team},
\ref{ch:robot-team-supervised}, \ref{ch:robot-team-gtm}, and
\ref{ch:robot-team-gsm} can be further developed in a number of interesting
directions. Of practical importance is the miniaturization of the UAV platform.
Many new hyperspectral imagers are now in development which weigh considerably less
than the Resonon Pika XC2 used in the current iteration of the UAV. For example, the
company Ximea has created a miniaturized hyperspectral imager called the \textit{xiSpec}
which samples up to 150 bands across visible and NIR wavelengths at a weight of
only 32 grams
(\url{https://www.ximea.com/en/usb3-vision-camera/hyperspectral-usb3-cameras-mini},
last accessed 2024-09-12). This imager is also in a pushbroom configuration but uses a
\textit{wedge} filter installed directly onto the sensor to select wavelength
bands. This means that the HSI processing pipeline developed in
Chapter~\ref{ch:robot-team} could be applied to this system with little
modification. Additionally, the dramatic weight reduction enabled by smaller HSI
would make it possible to use cheaper and lighter UAV platforms while freeing up
carrying capacity for increased battery capacity.

Another direction for potential improvement to the UAV would be the inclusion of
a \textit{Light Detection and Ranging} (LiDAR) module. Many autonomous robotic
systems employ LiDAR for intelligent object avoidance. However, a nadir-oriented
LiDAR module could be used to accurately measure the UAV-to-ground distance
during flight to improve georectification quality. Currently, the processing
method assumes a flat ground (reasonable for flights above water), however,
further accuracy could be obtained by directly measuring the distance from the
hyperspectral imager to each pixel location on the ground. New LiDAR modules
such as the Leddar Vu8 simultaneously measure distances for multiple
segments across a wide field of view and are small enough to incorporate onto a
drone \cite{marin2021study}. A similar sensor installed together with the
hyperspectral imager would therefore enable automatic generation of digital
elevation maps during UAV flight.

Based on the success of the models developed in
Chapter~\ref{ch:robot-team-supervised}, another obvious direction is to further
explore the capabilities of the robot team for direct estimation of water
composition parameters by out-fitting the USV with additional sensors for other
chemicals and ions of interest. In particular, the success of the machine
learning models for predicting ion concentrations should be further explored for
more ions at additional testing sites. Since ion concentrations are themselves not
an optically active component of the water, but rather, are related to the
specific mineral content and anthropogenic sources present at each water body,
models for ion concentrations will likely need to be trained for each separate
sensing location.

\section{GTM}

Chapter~\ref{ch:robot-team-gtm} demonstrated the value of using unsupervised
machine learning methods to visualize the distribution of HSI spectra. As
previously mentioned, one clear application of this technique is to guide the
collection of in situ data with the USV. In
Chapter~\ref{ch:robot-team-supervised}, the USV sampled the water using a
uniform grid across the entire pond. However, the distributions of chemical,
ionic, and biochemical components of the water often varied independently of one
another \textit{and} showed significant gradients at small spatial scales such
that a naive USV route could easily miss key areas of elevated concentrations.
By first training a GTM from HSI collected by an initial UAV flight, the USV
could then be intelligently guided to sample locations so as to maximize the
area explored in the GTM latent space rather than to maximize the spatial extent
explored in the pond. This would \textit{both} reduce time wasted sampling in
regions with uniform values \textit{and} guarantee that the USV would capture
sufficient data associated with the observed spectral variability.

Another potential application of the GTM is to the analysis of HSI from remote
sensing platforms. For example, harmful algal blooms (HAB) initiated by the
introduction of nutrients from fertilizers and other effluent runoff can be
distinguished from the surrounding water due to their specific spectral
characteristics \cite{algal-blooms-rs}. In addition to harming aquatic
ecosystems by blocking sunlight from penetrating into the water, some species
release toxic substances including neurotoxins and carcinogens
\cite{hab-toxins}. A variety of supervised machine learning methods have been
suggested to classify HAB from remote sensing imagery, but these approaches rely
on sparse ground-truth data \cite{hab-ml}. The GTM applied to remotely sensed HSI
naturally enables the segmentation of spectral signatures which could be used to
identify and distinguish HAB types.

\section{GSM}

The GSM was developed in Chapter~\ref{ch:robot-team-gsm} to model nonlinear
endmember mixing in HSI. Interestingly, this problem is identical to the problem
of \textit{source apportionment} in the air quality domain. Here one attempts to
analyze pollution measurement data such as PM composition
obtained via mass spectrometry to identify component profiles corresponding to unique
sources like fossil fuels, biomass burning, dusts, and pollen.
Traditionally, source apportionment analyses adopt linear receptor models which
assume that source profiles measured at a sensing site (the receptor) are the
same as those produced at their source \cite{source-apportionment-methods}. Such
linear models can not account for nonlinear effects from chemical reactions and
atmospheric transport along air parcel trajectories, but comprehensive chemical
transport models are often too coarse to capture variability at spatial scales
relevant to individual sensing sites. The original formulation of NMF (then
called \textit{positive} matrix factorization) was introduced by Paatero and
Tapper to model linear mixing for source apportionment \cite{pmf-orig,
  pmf-algs}. The GSM is therefore naturally suited for source apportionment
analyses where it can be used to obtain source signatures while accounting for
linear \textit{and} nonlinear mixing effects.

\section{PM Time Series Modeling}

The HAVOK model developed in Chapter~\ref{ch:havok} presents a promising method
for both PM time series forecasting \textit{and} the automatic detection of
outlier events corresponding to brief PM episodes. Comprehensive corrections to
historical PM measurements made across the MINTS air quality network are
currently in development to account for variability due to hygroscopic growth
during periods with high humidity or precipitation \cite{prabuddha-thesis}. Once
these corrections are incorporated, time series data from multiple sensors
across multiple years will be used to train new HAVOK models. The
forcing functions learned for each sensor will then be used to examine historical data
to identify the quantity of intermittent pollution events and quantify their
impact on 24-hour averaged PM data. This is highly relevant as the EPA's PM
regulations stipulate maximum allowable values for PM data at a 24-hour average.
By comparing average values with and without these intermittent PM spikes, we
will be able to discern whether or not increasing PM trends are due to a general
increase in overall PM \textit{or} due to increased activities which contribute in
acute chunks. The outcome of this analysis could directly impact strategies
chosen to address growing air quality concerns.

With access to longer durations of continuous PM time series, HAVOK
models will also be trained at coarser time resolutions in order to extend the
forecasting horizon. The model built in Chapter~\ref{ch:havok} was able to
make successful forecasts for up to 5 minutes corresponding to 30 future time
steps. With a comparable volume of data at a 15 minute resolution, a 30 step
prediction would instead yield a 7.5 hour forecast. Enabling individual
forecasts of this type at each sensor location will dramatically increase the
utility of the distributed sensing network.

To model PM 2.5 time series data, we found that extending HAVOK to
allow for \textit{multiple} forcing functions resulted in improved performance.
Future work will explore further extensions to the HAVOK model to combine data from
\textit{multiple} time series, for example, the time series for each PM size
fraction measured by the IPS7100 sensor. Provided that all time series are
temporally aligned and sampled at the same rate, Hankel matrices formed from the time
delay embeddings of each time series can be computed and then concatenated into
a single embedding matrix. This \textit{combined} matrix can then be
decomposed using the SVD before fitting a single HAVOK model. Such an approach
could simultaneously capture the dynamics of PM across multiple size scales.

\section{Air Parcel Back Trajectories}

Equally important to identifying source signatures and forecasting future
values is the ability to spatially localize PM sources. Wind velocity
data sourced from meteorological agencies like the European Center for
Medium-Range Weather Forecasts (ECMWF) and the US National Oceanic and
Atmospheric Administration (NOAA) can be used to integrate air parcel trajectories
backwards in time from a sensor site in order to locate likely PM sources. To
accomplish, this we are currently developing a parallelized Lagrangian particle
tracking code to complement the models developed in this dissertation. The code
will be evaluated for accuracy against NOAA's Hysplit model, but importantly,
will be designed to directly integrate with the MINTS air quality network from
Chapter~\ref{ch:air-network} to compute back trajectories for live PM data

