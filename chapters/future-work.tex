\chapter{Future Work}\label{ch:future-work}

\section{Robot Team}

Minaturization of drones - next generation hyperspectral imagers (smaller) on smaller drones with additional payloads for visible and thermal imagers (not previously utilized)

Multi-sensor fusion and super resolution, e.g. use fine spatial resultion of visible imager together with spectral resolution of HSI and thermal to create a blended data product

Combine drone based imaging with remote sensing hyperspectral data, e.g. Enmap, PACE, etc...

Drone swarms for faster data collection

Additional data collection (can we test the generalizability across water bodies)

Additional ions with application to battery stuff...

\section{GTM}

GTM for guided data collection viat prize-collecting travelling salesman problem.

GTM for dust source identification. We can encourage finer cluserting behavior by augmenting the GTM to use adaptive mixing coefficients $\pi_k$ as we did for the GSM.

Batch version implementation of the GTM for \textit{big} datasets and online learning.

\section{GSM}

GSM for algal bloom identifiaction using remotely sensed hyperspectral imagery.

GSM for source apportionment of air quality measurements

Batch version of the GSM for \textit{big} dtaasets.


\section{PM Modeling}

Using HAVOK model to automatically identify pollution events and generate short term predictions near real time.

Exploring other physics informed models for analyzing PM data, e.g. Lagrangian
and Hamiltonian Neural Networks as another potential avenue -- Train on short
time scales and identify deviations of level surfaces of the Hamiltonian as a
means for identifying \textit{external forcing} like effects.

\section{Air Parcel Back Trajectories}

Compute back trajectories of PM data using meteorological analysis (and re-analyses).

Applications to human health outcomes: Alzheimers


\section{Chemical Data Assimilation}

Air quality chamber.





