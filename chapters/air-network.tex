\chapter{Distributed Sensor Networks for Air Quality Assessment}\label{ch:air-network}



\section{Particulate Matter and Air Quality}

\subsection{What is Particulate Matter}

\subsection{Environmental Impact}

\textcolor{red}{include a picture of fog/haze}

\subsection{Human Impact}


\subsection{Measurement Methods}

Describe federal method (gravimetric analysis) and optical particle counter
designs based on Mie theory.

\subsection{Problems of Scale}

Describe why we need dense sensor networks, specifically to address the relevant
spatio-temporal scales that are missed by averaging to suggested EPA standards.

Cite Prabuddha's paper and discuss value of dense sensor networks for providing
in-situ reference data which can be used to calibrate remote sensing data
products - highlight the application of PACE e.g. for black-carbon.

\section{Low Cost Sensors for Real-Time PM Monitoring}

\subsection{Sensor Design}

include discussion of limitations, e.g. known problem of hygroscopic growth,
which is why we incorporate a battery of sensors including sensors for
meteorological parameters which can be used to post-process PM data for humidity correction.

\subsection{Sensor Networking}

\section{Towards Real-Time Assessment and Data-Driven Decisions}


\subsection{The Data Pipeline}

Describe docker and containerization. Goal: simple, maintainable framework which
can easily scale as additional sensors are incorporated.

The entire pipeline should be easily reproducible to enable local development
and make it straight forward to transition to cloud-based computing via Amazon
EC2 and other cloud services.

We want


\subsection{Live Dashboards}


\subsection{Automated Reports}

Highlight reproducibility as a key analysis criterion


