\chapter{Unsupervised Characterization of Water Composition with Generative Topographic Mapping}\label{ch:robot-team-gtm}

\textcolor{red}{
  Unmanned aerial vehicles equipped with hyperspectral imagers have emerged as
  an essential technology for the characterization of inland water bodies. The
  high spectral and spatial resolutions of these systems enable the retrieval
  of a plethora of optically active water quality parameters via band ratio
  algorithms and machine learning methods. However, fitting and validating
  these models requires access to sufficient quantities of in situ reference
  data which are time-consuming and expensive to obtain. In this study, we
  demonstrate how Generative Topographic Mapping (GTM), a probabilistic
  realization of the self-organizing map, can be used to visualize
  high-dimensional hyperspectral imagery and extract spectral signatures
  corresponding to unique endmembers present in the water.  Using data
  collected across a North Texas pond, we first apply  GTM to visualize the
  distribution of captured reflectance spectra, revealing the small-scale
  spatial variability of the water composition. Next, we demonstrate how the
  nodes of the fitted GTM can be interpreted as unique spectral endmembers.
  Using extracted endmembers together with the normalized spectral similarity
  score, we are able to efficiently map the abundance of nearshore algae, as
  well as the evolution of a rhodamine tracer dye used to simulate water
  contamination by a localized source.
}



\section{Motivation}

\section{Unsupervised Learning}

\section{Principal Component Analysis}

\section{Self Organizing Maps}

\section{Generative Topographic Mapping}

\section{Study Overview}

\section{Results}

\section{Discussion}
