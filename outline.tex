\documentclass{article}
\usepackage{graphicx}
\usepackage[margin=0.75in]{geometry}

\usepackage{outlines}
\usepackage{amsmath}
\usepackage{xcolor}

\usepackage[authoryear]{natbib}
\bibliographystyle{chicago}

\begin{document}

\section{Introduction}
\begin{outline}[enumerate]
  \1 Opening Paragraph
      \2 The rapid pace of global change increasingly endangers human health and
      well-being
      and threatens environmental health and stability
      \2 To face this threat, we need to make \textit{data-driven decisions}.
      This requires both comprehensive data and the means to evaluate/interpret
      them.
      \2 Continued advancements in low-cost, mobile sensing have dramatically
      increased our ability to characterize the environment by providing in-situ
      data at spatial and temporal resolutions relevant to human-scale
      interactions \textcolor{red}{get scales from Dr. Lary's previous papers}.
      \2 Additionally, a plethora of satellite missions have been recently
      launched (or will soon be) which promise to provide petabytes of
      multi-spectral imagery across the globe
      \2 However, the data provided by these systems are often limited to specific
      physical quantities (e.g. reflectances) which are only indirectly related
      to water quality and air quality parameters which we actually care about
      \2 To make sense of these data, one can apply physical models which relate
      measurements to parameters of interest
      \2 Even if we suppose that we have perfectly characterized the relevant
      physics, simulating the dynamics of physical systems at scale
      demands significant computational resources limiting real-time
      applicability.
      \2 For example, ECMWF meteorological forecasts are provided a (add spatial
      and temporal resolution) despite the fact the many studies across the
      scientific literature suggest relevant spatial scales between 0-1 km and
      temporal scales at the order of 10 seconds for air quality.
      \2 Machine learning offers a promising solution to this problem: take
      advantage of increasingly large datasets to extract actionable insights
      \2 However, most machine learning problems are developed to work on
      general (abstract) tasks such as image classification
      \2 It is wasteful to force the ML models to re-learn physics from scratch.
      Rather, we seek to incorporate our prior physics knowledge to let the ML
      methods extract the \textit{missing pieces} from the comprehensive data we
      collect, i.e. to fill in the gaps.

  \1 Dissertation Goals
  \1 Motivation
      \2 Water Quality
      \2 Air Quality
      \2 Indoor Air Quality
  \1 Physical Sensing
  \1 The role of Machine Learning in the Era of Big Data
  \1 Physics-based Machine Learning
  \1 Actionable Insights and Societal Value
  \1 Dissertation Overview
  \1 Summary of Works Produced
      \2 First Author Papers
      \2 Collaborative Work
      \2 Code Repositories
      \2 Datasets Generated
\end{outline}


\section{Physical Sensing for Environmental Quality Assessment}

\section{Robot Team 1: Supervised}

\section{Robot Team 2: GTM}

\section{Robot Team 3: GSM}

\section{Sensor Network}

\section{HAVOK}

\section{AutoChem}

\section{Future Work}

\section{Conclusions}


\end{document}



\bibliography{./references.bib}

