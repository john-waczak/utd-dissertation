The goal of this thesis is advancing physical sensing in service of society to provide actionable insights. This goal is pursued by applying physics informed approaches together with a suite of sensing and computational technologies, implementing the reusable paradigm of software defined sensors, i.e. physical sensing elements wrapped in a software layer. This software layer can serve a variety of purposes such as calibration and the provision of enhanced or derived data products. It is part of a broader effort in the MINTS-AI laboratory at the University of Texas at Dallas. Where MINTS-AI is an acronym, \underline{M}ulti-Scale \underline{M}ulti-Use \underline{Int}egrated \underline{Int}elligent \underline{Int}eractive \underline{S}ensing in \underline{S}ervice of \underline{S}ociety for \underline{A}ctionable \underline{I}nsights.

Comprehensive environmental sensing is a timely and beneficial endeavor for a variety of reasons. The growing awareness of major environmental issues such as climate change, pollution, and habitat loss necessitates effective environmental monitoring and management. Comprehensive environmental sensing can provide real-time data on air and water quality, weather patterns, and other environmental factors, assisting in the identification and resolution of environmental issues. This assists in the development and implementation of policies and strategies aimed at reducing environmental impact and increasing sustainability. Given that, for instance, air quality can have significant effects on human health, this has particular societal value. 

Exposure to air pollution has been linked to a wide range of health effects \citep{Brook2008, Kelly2011, Xu2017}, including respiratory and cardiovascular diseases, cancer, and adverse birth outcomes. Further, physical sensing provides valuable data and the basis for international assessments such as the Intergovernmental Panel on Climate Change (IPCC), which seeks to assess the science related to climate change and its impacts on natural and human systems \citep{IPCC1990, IPCC1995, IPCC2001, IPCC2007a, IPCC2007b, IPCC2007c, IPCC2013a, IPCC2013b, IPCC2014, IPCC2018, Friedlingstein2020, Huang2017}. 

Comprehensive sensing of the environment can improve decision-making. The real-time and accurate data provided by environmental sensors can aid in informed decision-making regarding various aspects such as traffic management, industrial regulation, and crop planning. For instance, data on air quality can be used to inform decisions about reducing pollution levels, while data on weather patterns can help farmers to plan their crops and reduce water usage. Comprehensive sensing of the environment can be instrumental in emergency response. Real-time data on weather patterns, air quality, water levels and resources, and seismic activity can help emergency responders to prepare for and respond to natural disasters such as hurricanes, floods, and earthquakes. The quick and accurate information can enable effective and timely response, potentially saving lives and reducing the impact of the disaster. 

Many advances in technology have enabled the creation of comprehensive sensing systems that can monitor and analyze data from various sensors and devices in real-time. In this thesis we use a range of technologies including autonomous robotic teams \citep{Dunbabin2012, Rubenstein2014, Chen2017}, hyperspectral imaging \citep{Plaza2009, Li2018, Zhu2017}, mesh networks utilizing the Internet of Things (IoT) \citep{Gubbi2013, Atzori2010, Al-Fuqaha2015}, machine learning (ML) \citep{Goodfellow2016, LeCun2015, Jordan2015}, edge computing, high-performance computing,  wearable sensors and modern high-performance dynamic programming languages such as Julia \citep{Bezanson2017} designed for numerical and scientific computing. These technologies have facilitated the collection and processing of large amounts of data from multiple sources, resulting in more accurate and comprehensive environmental monitoring.

