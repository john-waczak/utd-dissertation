\documentclass[doublespacing]{utdthesis}
% For one-and-a-half spacing, use: \documentclass[halfspacing]{utdthesis}

%%% Load any desired packages in the space below.
%%% Warning: Do not load packages that change the margins, headers, or footers!
%%%
% Optional: If you want to use Times as your font, load it here.  Note that
% although package "times" should work, it may not be the best choice.  Newer
% LaTeX distributions offer "mathptmx" and "newtxtext,newtxmath" as superior
% replacements.  You should find out which is best for your LaTeX.  (If this
% sounds confusing, you probably shouldn't try to change the font to Times.)
%\usepackage{times}
%
% Optional: If your LaTeX has microtype, use it to improve text quality:
\usepackage{microtype}
%
% Recommended: If your dissertation contains math, use the AMS packages:
\usepackage{amsmath,amssymb,amsthm}
%
% Recommended: If your dissertation needs embedded graphics, use graphicx:
\usepackage{graphicx}
%
% Recommended: If your bibliography contains web page URLs, the url package
% improves their appearance (e.g., better line breaking):
\usepackage{url}
%
% Required: To satisfy UTD's formatting requirements for citations, use the
% "natbib" package.  (Use other citation packages at your own risk; not all
% are flexible enough to meet UTD's requirements.)  If you wish to use numeric
% citations, change "authoryear" to "numbers" below.  Use the "chicago" BibTeX
% style, which most closely matches the Turabian formatting required by UTD.
% UTD mandates a blank line between each pair of bibliography entries, so set
% \bibsep as shown below.  Finally, if you are accustomed to using \cite as
% your citation macro, point it to natbib's \citep macro as shown.
\usepackage[authoryear]{natbib}
\bibliographystyle{chicago}
\setlength{\bibsep}{12pt plus 1pt minus 1pt}
\let\cite=\citep
%
% Required: If you have any wide tables or figures that need to be typeset
% in landscape, use the rotating package:
\usepackage{rotating}
%
% Optional: If you use hyperref to auto-generate hyperlinks, always load it
% LAST since it modifies everything else.  In addition, only load hyperref if
% you use pdftex or pdflatex to generate PDFs directly.  Do NOT use it if you
% use plain tex or latex to generate a DVI file.  (If you are generating DVI
% files which you then convert to PDF, you should seriously consider switching
% to pdflatex.  The DVI format loses information because it cannot support
% modern PDF document features.  Using pdflatex to generate PDFs directly
% therefore results in documents of significantly higher quality.)
\usepackage{ifpdf}
\ifpdf
  \usepackage{hyperref}
\fi
%
%%% End of packages.

%%% Define all your personal macros here (if you have any).
%
\providecommand{\hyperref}[2][]{#2}

\newenvironment{exampleclasscode}
 {\parindent=1cm\vskip0pt plus2pt minus0pt\begin{verse}}
 {\end{verse}\vskip0pt plus2pt minus0pt}
%
%%% End of personal macro definitions.


%%% The following definitions MUST come before the document begins.
%
\author{John L Waczak}
\title{Title \\ Physical Sensing Coupled with Physics Based Machine Learning}
\thesistype{Dissertation}  % or "Thesis"
\degreefull{Doctor of Philosophy}
\degreeabbr{PhD}
\subject{Physics}
\graduationmonth{May}
\graduationyear{2023}
\prevdegrees{BS} % comma-separated list of PREVIOUS degrees

% List committee members in order.  Mark chairpersons with a "*":
\committeemember*{David Lary}
\committeemember{Person 2}
\committeemember{Person 3}
\committeemember{Person 4}
%
%%% End of definitions.


%%% Beginning of actual thesis document.

\begin{document}

\frontmatter

\signaturepage
\copyrightpage{2012} % optional

\begin{dedication} % optional
This thesis class file \\
is dedicated to my students, \\
who suffered without a proper one \\
until the present time.
\end{dedication}

\maketitle

\begin{acks}{December 2022} % date when thesis first submitted to committee
  Update required!
\end{acks}

\begin{abstract}
  Update required!  
\end{abstract}

\tableofcontents
\listoffigures % required if you have any figures
\listoftables % required if you have any tables

\mainmatter

\chapter{Introduction}
\label{c:introduction}
The goal of this thesis is advancing physical sensing in service of society to provide actionable insights. This goal is pursued by applying physics informed approaches together with a suite of sensing and computational technologies, implementing the reusable paradigm of software defined sensors, i.e. physical sensing elements wrapped in a software layer. This software layer can serve a variety of purposes such as calibration and the provision of enhanced or derived data products. It is part of a broader effort in the MINTS-AI laboratory at the University of Texas at Dallas. Where MINTS-AI is an acronym, \underline{M}ulti-Scale \underline{M}ulti-Use \underline{Int}egrated \underline{Int}elligent \underline{Int}eractive \underline{S}ensing in \underline{S}ervice of \underline{S}ociety for \underline{A}ctionable \underline{I}nsights.

Comprehensive environmental sensing is a timely and beneficial endeavor for a variety of reasons. The growing awareness of major environmental issues such as climate change, pollution, and habitat loss necessitates effective environmental monitoring and management. Comprehensive environmental sensing can provide real-time data on air and water quality, weather patterns, and other environmental factors, assisting in the identification and resolution of environmental issues. This assists in the development and implementation of policies and strategies aimed at reducing environmental impact and increasing sustainability. Given that, for instance, air quality can have significant effects on human health, this has particular societal value. 

Exposure to air pollution has been linked to a wide range of health effects \citep{Brook2008, Kelly2011, Xu2017}, including respiratory and cardiovascular diseases, cancer, and adverse birth outcomes. Further, physical sensing provides valuable data and the basis for international assessments such as the Intergovernmental Panel on Climate Change (IPCC), which seeks to assess the science related to climate change and its impacts on natural and human systems \citep{IPCC1990, IPCC1995, IPCC2001, IPCC2007a, IPCC2007b, IPCC2007c, IPCC2013a, IPCC2013b, IPCC2014, IPCC2018, Friedlingstein2020, Huang2017}. 

Comprehensive sensing of the environment can improve decision-making. The real-time and accurate data provided by environmental sensors can aid in informed decision-making regarding various aspects such as traffic management, industrial regulation, and crop planning. For instance, data on air quality can be used to inform decisions about reducing pollution levels, while data on weather patterns can help farmers to plan their crops and reduce water usage. Comprehensive sensing of the environment can be instrumental in emergency response. Real-time data on weather patterns, air quality, water levels and resources, and seismic activity can help emergency responders to prepare for and respond to natural disasters such as hurricanes, floods, and earthquakes. The quick and accurate information can enable effective and timely response, potentially saving lives and reducing the impact of the disaster. 

Many advances in technology have enabled the creation of comprehensive sensing systems that can monitor and analyze data from various sensors and devices in real-time. In this thesis we use a range of technologies including autonomous robotic teams \citep{Dunbabin2012, Rubenstein2014, Chen2017}, hyperspectral imaging \citep{Plaza2009, Li2018, Zhu2017}, mesh networks utilizing the Internet of Things (IoT) \citep{Gubbi2013, Atzori2010, Al-Fuqaha2015}, machine learning (ML) \citep{Goodfellow2016, LeCun2015, Jordan2015}, edge computing, high-performance computing,  wearable sensors and modern high-performance dynamic programming languages such as Julia \citep{Bezanson2017} designed for numerical and scientific computing. These technologies have facilitated the collection and processing of large amounts of data from multiple sources, resulting in more accurate and comprehensive environmental monitoring.



\chapter{Global Change}
\label{c:GlobalChange}
Global change refers to the significant and long-term alterations in the Earth's physical, chemical, and biological systems, resulting from natural and human-induced processes \citep{IPCC2014, IPCC2018, UN2015}. This includes changes in the climate, land use, biodiversity, and biogeochemical cycles, as well as interactions among these systems. Global change can have profound impacts on natural and human systems, including altered weather patterns, sea level rise, increased frequency and severity of extreme events, loss of biodiversity and ecosystem services, and effects on human health and well-being. Understanding and managing global change is a critical challenge facing society today, requiring interdisciplinary approaches and collaboration across sectors and regions.

Global change can have a range of impacts on society, including environmental, social, and economic effects. Some of the aspects of global change that have the biggest impact on society include:

\begin{enumerate}
    \item Climate Change: Climate change, driven by human activities such as burning fossil fuels, deforestation, and land-use changes, has impacts on natural systems such as ocean acidification, sea level rise, and changes in precipitation patterns. These impacts can have cascading effects on human systems, including impacts on food security, water availability, and health.
    \item Biodiversity Loss: Global change can lead to the loss of biodiversity, which can have impacts on ecosystem functioning and services, such as pollination, pest control, and carbon storage. These impacts can have indirect effects on human well-being, including impacts on food security, health, and cultural heritage.
    \item Land Use Change: Land use change, such as deforestation, urbanization, and agriculture, can have impacts on natural systems such as soil quality, water availability, and biodiversity. These impacts can have direct and indirect effects on human systems, including impacts on food security, water availability, and cultural heritage.
    \item Economic and Social Inequality: Global change can exacerbate economic and social inequality, with impacts on access to resources, health, and well-being. These impacts can have cascading effects on the ability of societies to adapt and respond to global change.
    \item Human Health: Global change can have significant impacts on human health \citep{WHO2018, Costello2009, Haines2006}, both directly and indirectly, for example:
    \begin{enumerate}
        \item Heat-related Illness: As temperatures increase due to global warming, there is an increased risk of heat-related illness, including heat exhaustion and heat stroke, particularly in vulnerable populations such as the elderly, young children, and outdoor workers.
        \item Air Pollution: Global change can lead to increased air pollution, including from sources such as wildfires and fossil fuel combustion. Exposure to air pollution can increase the risk of respiratory and cardiovascular diseases, including asthma, chronic obstructive pulmonary disease (COPD), and heart disease.
        \item Vector-borne Diseases: Changes in temperature and precipitation patterns can affect the distribution and abundance of disease vectors such as mosquitoes and ticks, leading to increased risks of vector-borne diseases such as dengue fever, malaria, and Lyme disease.
        \item Waterborne Diseases: Changes in precipitation patterns and water quality can increase the risk of waterborne diseases, including cholera and other diarrheal diseases.
        \item Food Security: Global change can affect food production and availability, leading to food shortages and malnutrition, particularly in vulnerable populations such as children and pregnant women.
    \end{enumerate}
\end{enumerate}

Effectively addressing these aspects of global change requires interdisciplinary approaches and collaboration across sectors and regions, as well as a commitment to sustainable development and equitable solutions. Adaptation and mitigation are two strategies for addressing global change, which differ in their focus and approach.

Adaptation involves taking measures to adjust and respond to the impacts of global change that are already occurring or are expected to occur in the future. This can include actions such as building sea walls to protect against sea level rise, developing drought-resistant crops, and improving public health infrastructure to address the increased risk of vector-borne diseases. Adaptation strategies aim to reduce the vulnerability of human and natural systems to the impacts of global change and increase their resilience.

Mitigation involves taking measures to reduce the drivers of global change, such as greenhouse gas emissions, land use change, and deforestation. This can include actions such as increasing energy efficiency, shifting to renewable energy sources, and reducing waste and consumption. Mitigation strategies aim to address the root causes of global change and reduce its severity and impact.

Both adaptation and mitigation are important strategies for addressing global change, but they differ in their focus and approach. Adaptation strategies focus on responding to the impacts of global change that are already occurring or are expected to occur in the future, while mitigation strategies focus on reducing the drivers of global change and preventing its impacts from occurring in the first place. A comprehensive approach to global change will require both adaptation and mitigation strategies, as well as efforts to promote sustainable development and equitable solutions.

\section{The Role of Sensing}
Sensing technologies can play a critical role in both adaptation and mitigation efforts by providing data and information that can inform decision-making and improve the effectiveness of strategies \citep{UNEP2017, NRC2010, CEN2017}.

In adaptation efforts, sensing technologies can provide real-time data on environmental conditions such as temperature, precipitation, sea level, air quality, as well as on the status and health of ecosystems and wildlife. This information can be used to inform early warning systems for natural disasters, to track the spread of vector-borne diseases, and to monitor the impacts of climate change on biodiversity and ecosystem services. Sensing technologies can also provide data on the effectiveness of adaptation measures, such as the performance of sea walls and other infrastructure.

In mitigation efforts, sensing technologies can provide data on greenhouse gas emissions and other drivers of global change, as well as on the effectiveness of mitigation measures such as renewable energy and carbon capture and storage. Sensing technologies can also be used to monitor and manage land use changes such as deforestation and urbanization, and to track the impacts of these changes on ecosystems and carbon storage.

Overall, sensing technologies can provide critical data and information for both adaptation and mitigation efforts, helping to improve decision-making and increase the effectiveness of strategies. The integration of sensing technologies with other tools such as modeling and data analysis can also help to identify new strategies and solutions for addressing global change. There are various sensing technologies and approaches used for monitoring the global environment. Here are some of the key ones:

\begin{enumerate}
    \item Remote Sensing: This technology involves using satellites and other airborne platforms to collect data on the Earth's atmosphere, land, and oceans. Remote sensing provides information on environmental parameters such as temperature, humidity, air quality, land use and land cover, and ocean temperature, salinity, and sea level \citep{Thenkabail2019, Buyantuyev2017, Gamon2016, Wang2017, Pasher2019}. Some examples of remote sensing include:
    \begin{enumerate}
        \item Lidar: This technology uses laser pulses to measure distance and can be used to create detailed three-dimensional maps of the environment. Lidar is commonly used to measure forest canopy height, but can also be used to measure atmospheric conditions such as cloud cover and aerosol concentrations.
        \item Imaging Spectroscopy: This technology uses a combination of imaging and spectroscopy to measure the reflectance of different wavelengths of light. Imaging spectroscopy can be used to identify and map different types of vegetation and minerals, and can provide information on the health of plant communities.
        \item Unmanned Aerial Vehicles (UAVs): These are remote-controlled or autonomous aircraft that can be equipped with sensors for remote and in-situ  environmental monitoring. UAVs can be used for mapping and monitoring of large areas, and can collect high-resolution data on environmental conditions.
    \end{enumerate}
    \item In-Situ Sensors: These sensors are used to collect data directly from the environment at the location of interest. They can measure environmental parameters such as temperature, pressure, and humidity, as well as water quality and soil moisture. In situ sensors are commonly used in marine environments to measure ocean temperature, salinity, and other properties. Some examples of in-situ sensing include:
    \begin{enumerate}
        \item Weather Stations: These are automated weather monitoring systems that collect data on atmospheric conditions such as temperature, humidity, barometric pressure, wind speed and direction, and precipitation. Weather stations can be installed on land or in the ocean to provide continuous monitoring of environmental conditions.
        \item Ground-Based Sensors: These sensors are used to monitor the quality of air, water, and soil. They can detect and measure pollutants such as carbon dioxide, nitrogen dioxide, ozone, sulfur dioxide, and particulate matter. Ground-based sensors are installed in various locations such as cities, industrial sites, and rural areas to provide localized environmental monitoring.
        \item Acoustic Sensors: These sensors are used to monitor environmental noise levels, including noise from traffic, industrial sources, and natural sources such as wind and waves. Acoustic sensors can provide information on noise levels over time and across different locations.
    \end{enumerate}
\end{enumerate}

\noindent Overall, these sensing technologies play a critical role in monitoring the global environment and can provide valuable information for environmental research, management, and policy-making.

\section{The Role of Computational Modelling}

Computer modeling can play a valuable role in both understanding and predicting global change \citep{Chen2019, Hantson2016, DeLucia2021, Oleson2013, Clark2016}. For example:

\begin{enumerate}
    \item Climate Modeling: Computer models can be used to simulate the Earth's climate system and predict future climate conditions. These models can incorporate data on greenhouse gas emissions, land use changes, and other factors to project how the Earth's climate will change over time.
    \item Ecosystem Modeling: Computer models can be used to simulate how ecosystems will respond to changes in environmental conditions, such as changes in temperature, precipitation, and atmospheric composition. These models can help predict how changes in ecosystems will impact biodiversity, ecosystem services, and human well-being.
    \item Carbon Cycle Modeling: Computer models can be used to simulate the global carbon cycle, which is the exchange of carbon between the Earth's atmosphere, land, and oceans. These models can help predict how changes in carbon emissions and land use will impact atmospheric carbon dioxide concentrations and global climate.
    \item Air Quality Modeling: Computer models can be used to simulate air quality, including the dispersion of pollutants in the atmosphere. These models can help predict how changes in emissions and atmospheric conditions will impact air quality and human health.
    \item Hydrological Modeling: Computer models can be used to simulate the movement of water through the Earth's hydrological cycle. These models can help predict how changes in precipitation, land use, and other factors will impact water availability, quality, and distribution.
\end{enumerate}

Overall, computer modeling can provide valuable insights into the complex processes and interactions that drive global change. These insights can inform policy decisions and help guide efforts to mitigate and adapt to the impacts of global change.


\chapter{Key Technologies}
\label{c:KeyTechnologies}
\section{An Introduction to Some Key Tools Used in This Thesis }

\subsection{Julia for Scientific Computing}

Julia is designed to combine the ease of use and high-level abstractions of languages like Python with the performance of compiled languages like C++, achieving a unique combination of speed and productivity for numerical and scientific computing.  Julia is a high-level, high-performance programming language designed for numerical and scientific computing. It combines the ease of use and readability of Python with the speed and efficiency of Fortran or C.   Julia has a wide array of scientific computing functionality, making it a powerful language for numerical analysis, data science, and engineering. It has built-in support for arrays and linear algebra, as well as packages for differential equations, optimization, probabilistic programming, data analysis and visualization, parallel and distributed computing, and machine learning. Julia's combination of performance, expressiveness, and flexibility make it an excellent choice for scientific and engineering applications, allowing for high-level abstractions and rapid prototyping, while still providing low-level control and efficient execution.

Here are some examples of what can be done easily in Julia that may not be as easy or efficient in other widely used scientific computing languages such as Python or Fortran:

\begin{enumerate}
    \item Multiple dispatch: Julia has a powerful multiple dispatch system that allows for generic programming and efficient function overloading. This allows for more flexible and expressive code compared to traditional object-oriented programming (OOP) in Python. Multiple dispatch allows a function to behave differently based on the types and/or number of arguments passed to it. In other words, the behavior of a function can be \lq dispatched' based on the specific types and/or number of arguments passed to it.
    \item Just-in-time (JIT) compilation: Julia's JIT compiler translates high-level Julia code into optimized machine code, making Julia programs run nearly as fast as C or Fortran. In contrast, Python code is interpreted, and Fortran requires pre-compilation.
    \item Distributed computing: Julia has built-in support for distributed computing, making it easy to parallelize and scale up computations across multiple processors or machines. This is not as easy to do in Python or Fortran.
    \item Units and Error Propagation: The Units package in Julia provides a powerful and flexible framework for handling physical units in computations, useful for error propagation and dimensional analysis, helping to ensure that the results are accurate, consistent, easy to interpret, and dimensionally consistent. 
    \item Built-in unit testing: Julia has a built-in testing framework that makes it easy to write and run unit tests for your code, ensuring that it works correctly.
    \item ISO characters: Julia supports the use of Greek and other ISO characters in variable and function names, which can make code more readable and expressive, especially in mathematical or scientific contexts.
    \item Interactive data visualization: Julia has a number of powerful data visualization packages, such as Plots.jl and Makie.jl, that allow for interactive, high-performance data visualization.
    \item Package management: Julia has a sophisticated package manager that makes it easy to install, manage, and use third-party packages in your code. This is not as easy to do in Fortran, and while Python has a package manager, Julia's package manager is faster and more reliable.
    \item Inline C/Fortran/Python/R/Matlab code: Julia allows for inline C, Fortran, Python, R or Matlab code, making it easy to use existing libraries and code written in these languages without having to rewrite everything in Julia.
\end{enumerate}


\subsection{Scientific and Physics-based Machine Learning}

Scientific machine learning (SciML) refers to the application of Machine Learning (ML) techniques to scientific problems, where the goal is not only to make predictions but also to gain insights into the underlying physical processes \citep{raissi2019physics, rackauckas2020universal, carleo2019machine}. SciML involves the integration of domain-specific knowledge and physical models with data-driven techniques, and it has the potential to revolutionize many areas of science and engineering. In this thesis we explore the use of Physics-based machine learning (PBML) \citep{Raissi2021, Wu2021} for a variety of applications. 

Recent examples include a paper by \citet{raissi2019physics} that introduces a physics-informed neural network (PINN) framework for solving nonlinear partial differential equations, a paper by \citet{rackauckas2020universal} that proposes a universal differential equation (UDE) approach to scientific machine learning, and a review article by \citet{carleo2019machine} that discusses the use of machine learning in various fields of physics, including condensed matter physics, high-energy physics, and quantum physics. PBML has several advantages over purely data-driven approaches, including:

\begin{enumerate}
    \item Improved generalization: PBML models incorporate prior knowledge of the underlying physics, resulting in models that are more interpretable and generalizable. This enables the models to make accurate predictions even with limited training data.
    \item Incorporation of physical constraints: PBML models can be designed to incorporate physical constraints, such as conservation laws, which can help to ensure physically consistent predictions.
    \item Improved interpretability: PBML models are more interpretable than purely data-driven models since they are designed to incorporate physical principles. This can enable scientists and engineers to gain deeper insights into the underlying mechanisms of the systems they are studying.
    \item Reduced data requirements: PBML models require less training data than purely data-driven models since they leverage the physics-based priors, reducing the need for large datasets to train accurate models.
    \item Better extrapolation: PBML models are better equipped to extrapolate beyond the training data since they incorporate knowledge of the underlying physics, enabling them to make more accurate predictions in new and unseen scenarios.
\end{enumerate}
\noindent Overall, PBML has several advantages over purely data-driven approaches, including improved generalization, reduced data requirements, better extrapolation, incorporation of physical constraints, and improved interpretability, making it a valuable tool for scientific and engineering applications.



\chapter{Autonomous sensing}
\label{c:sensing}
\section{Autonomous Hyperspectral Imaging}
\label{s:hsi}
\section{Walking Robot and Hovercraft}
\label{s:wr}

\chapter{Super Resolution}
\label{c:super resolution}
\section{Hyper Spectral Images}
\label{s:sr with hsi}
\section{Visible Images}
\label{s:sr with visible}
\section{Thermal Images}
\label{s:sr with thermal}
\section{Temporal Super Resolution: Imputation}
\label{s:sr with time}

\chapter{Atmospheric Sensing}
\label{c:atmospheric sensing}
\section{Time Series Analysis for Network Nodes}
\label{s:time series}


\chapter{Biometric Analysis}
\label{c:biometrics}
\section{Pose Analysis}
\label{s:pose}
\section{Facial Landmark Analysis}
\label{s:face}

\chapter{Audio Event Analysis}
\label{c: Audio}
\section{Gunshot Detection}
\label{s:guns}
\section{Species Identification}
\label{s:birds}

\chapter{Topological Data Analysis in Physical Measurement}
\label{c:tda}
\section{Persistent Homology}
\label{s:homology}
\section{Time Series Visibility Graphs}
\label{s:visibility}
\section{Graph Spectrum Analysis}
\label{s:spectrum}
\section{Graph Neural Networks}
\label{s:gnn}

\chapter{Scientific Machine Learning}
\label{c:sciml}
\section{Interpretable Machine Learning }
\label{s:interpretable}
\section{Sparse Identification of Nonlinear Dynamics}
\label{s:sindy}
\section{Physics Informed Neural Networks}
\label{s:pinn}
\section{SciML Applications}
\label{s:sciml applications}



\chapter{Conclusion}
\label{c:conclusion}

\appendix % required only if you have appendixes

%Begin the bibliography:

\begin{thesisbib}  % <--- THIS LINE IS REQUIRED!


  % If you use BibTeX, typically the only command between \begin{thesisbib}
  % and \end{thesisbib} is:
  %
\bibliography{references}
  %
  % (where "mybibfile" is the name of your .bib file).  In order to keep this
  % sample file self-contained, I've created my bibliography manually below,
  % but most people wouldn't want to do that.
\end{thesisbib}  % <-- THIS LINE IS REQUIRED!


\begin{biosketch}
Kevin W.~Hamlen began learning the basics of \LaTeX{} in the Fall of 2000 in
order to publish computer science journal articles as part of his
Ph.D.~candidacy at Cornell University.
By the completion of his degree in 2006, he had written thousands of lines of
\TeX{} code.

After completing his Ph.D., Dr.~Hamlen joined the faculty of the Computer
Science Department at The University of Texas at Dallas, and graduated his
first two Ph.D.~students (Micah Jones and Sunitha Ramanujam) in 2011.
By the graduation of his third student (Richard Wartell) in 2012, he had
concluded that a properly crafted \LaTeX{} class file for UTD theses was badly
needed to streamline future dissertation preparations.
He therefore created this one in December 2012.
\end{biosketch}


\begin{vita}  % <-- THIS LINE IS REQUIRED!
  % Replace the lines below with your CV using any formatting you wish,
  % or put nothing in this section and replace these pages with your CV
  % in the resulting PDF file.  (But you MUST include the \begin{vita}
  % and \end{vita} lines even if you intend to replace the pages, since
  % those lines are needed to put the Curriculum Vitae entry into the
  % Table of Contents.)
\end{vita}  % <-- THIS LINE IS REQUIRED!


\end{document}

