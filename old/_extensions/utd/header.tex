%%% Load any desired packages in the space below.
%%% Warning: Do not load packages that change the margins, headers, or footers!
%%%
% Optional: If you want to use Times as your font, load it here.  Note that
% although package "times" should work, it may not be the best choice.  Newer
% LaTeX distributions offer "mathptmx" and "newtxtext,newtxmath" as superior
% replacements.  You should find out which is best for your LaTeX.  (If this
% sounds confusing, you probably shouldn't try to change the font to Times.)
%\usepackage{times}
%
% Optional: If your LaTeX has microtype, use it to improve text quality:
\usepackage{microtype}
%
% Recommended: If your dissertation contains math, use the AMS packages:
\usepackage{amsmath,amssymb,amsthm}
%
% Recommended: If your dissertation needs embedded graphics, use graphicx:
\usepackage{graphicx}
%
% Recommended: If your bibliography contains web page URLs, the url package
% improves their appearance (e.g., better line breaking):
\usepackage{url}
%
% Required: To satisfy UTD's formatting requirements for citations, use the
% "natbib" package.  (Use other citation packages at your own risk; not all
% are flexible enough to meet UTD's requirements.)  If you wish to use numeric
% citations, change "authoryear" to "numbers" below.  Use the "chicago" BibTeX
% style, which most closely matches the Turabian formatting required by UTD.
% UTD mandates a blank line between each pair of bibliography entries, so set
% \bibsep as shown below.  Finally, if you are accustomed to using \cite as
% your citation macro, point it to natbib's \citep macro as shown.
\usepackage[authoryear]{natbib}
\bibliographystyle{chicago}
\setlength{\bibsep}{12pt plus 1pt minus 1pt}
\let\cite=\citep
%
% Required: If you have any wide tables or figures that need to be typeset
% in landscape, use the rotating package:
\usepackage{rotating}
%
% Optional: If you use hyperref to auto-generate hyperlinks, always load it
% LAST since it modifies everything else.  In addition, only load hyperref if
% you use pdftex or pdflatex to generate PDFs directly.  Do NOT use it if you
% use plain tex or latex to generate a DVI file.  (If you are generating DVI
% files which you then convert to PDF, you should seriously consider switching
% to pdflatex.  The DVI format loses information because it cannot support
% modern PDF document features.  Using pdflatex to generate PDFs directly
% therefore results in documents of significantly higher quality.)
\usepackage{ifpdf}
\ifpdf
  \usepackage{hyperref}
\fi
%
%%% End of packages.

%%% Define all your personal macros here (if you have any).
%
\providecommand{\hyperref}[2][]{#2}

\newenvironment{exampleclasscode}
 {\parindent=1cm\vskip0pt plus2pt minus0pt\begin{verse}}
 {\end{verse}\vskip0pt plus2pt minus0pt}
%
%%% End of personal macro definitions.


