% Options for packages loaded elsewhere
\PassOptionsToPackage{unicode}{hyperref}
\PassOptionsToPackage{hyphens}{url}
\PassOptionsToPackage{dvipsnames,svgnames,x11names}{xcolor}
%
\documentclass[
  letterpaper,
  DIV=11,
  numbers=noendperiod]{scrartcl}

\usepackage{amsmath,amssymb}
\usepackage{iftex}
\ifPDFTeX
  \usepackage[T1]{fontenc}
  \usepackage[utf8]{inputenc}
  \usepackage{textcomp} % provide euro and other symbols
\else % if luatex or xetex
  \usepackage{unicode-math}
  \defaultfontfeatures{Scale=MatchLowercase}
  \defaultfontfeatures[\rmfamily]{Ligatures=TeX,Scale=1}
\fi
\usepackage{lmodern}
\ifPDFTeX\else  
    % xetex/luatex font selection
\fi
% Use upquote if available, for straight quotes in verbatim environments
\IfFileExists{upquote.sty}{\usepackage{upquote}}{}
\IfFileExists{microtype.sty}{% use microtype if available
  \usepackage[]{microtype}
  \UseMicrotypeSet[protrusion]{basicmath} % disable protrusion for tt fonts
}{}
\makeatletter
\@ifundefined{KOMAClassName}{% if non-KOMA class
  \IfFileExists{parskip.sty}{%
    \usepackage{parskip}
  }{% else
    \setlength{\parindent}{0pt}
    \setlength{\parskip}{6pt plus 2pt minus 1pt}}
}{% if KOMA class
  \KOMAoptions{parskip=half}}
\makeatother
\usepackage{xcolor}
\setlength{\emergencystretch}{3em} % prevent overfull lines
\setcounter{secnumdepth}{-\maxdimen} % remove section numbering
% Make \paragraph and \subparagraph free-standing
\ifx\paragraph\undefined\else
  \let\oldparagraph\paragraph
  \renewcommand{\paragraph}[1]{\oldparagraph{#1}\mbox{}}
\fi
\ifx\subparagraph\undefined\else
  \let\oldsubparagraph\subparagraph
  \renewcommand{\subparagraph}[1]{\oldsubparagraph{#1}\mbox{}}
\fi


\providecommand{\tightlist}{%
  \setlength{\itemsep}{0pt}\setlength{\parskip}{0pt}}\usepackage{longtable,booktabs,array}
\usepackage{calc} % for calculating minipage widths
% Correct order of tables after \paragraph or \subparagraph
\usepackage{etoolbox}
\makeatletter
\patchcmd\longtable{\par}{\if@noskipsec\mbox{}\fi\par}{}{}
\makeatother
% Allow footnotes in longtable head/foot
\IfFileExists{footnotehyper.sty}{\usepackage{footnotehyper}}{\usepackage{footnote}}
\makesavenoteenv{longtable}
\usepackage{graphicx}
\makeatletter
\def\maxwidth{\ifdim\Gin@nat@width>\linewidth\linewidth\else\Gin@nat@width\fi}
\def\maxheight{\ifdim\Gin@nat@height>\textheight\textheight\else\Gin@nat@height\fi}
\makeatother
% Scale images if necessary, so that they will not overflow the page
% margins by default, and it is still possible to overwrite the defaults
% using explicit options in \includegraphics[width, height, ...]{}
\setkeys{Gin}{width=\maxwidth,height=\maxheight,keepaspectratio}
% Set default figure placement to htbp
\makeatletter
\def\fps@figure{htbp}
\makeatother

\KOMAoption{captions}{tableheading}
\makeatletter
\makeatother
\makeatletter
\makeatother
\makeatletter
\@ifpackageloaded{caption}{}{\usepackage{caption}}
\AtBeginDocument{%
\ifdefined\contentsname
  \renewcommand*\contentsname{Table of contents}
\else
  \newcommand\contentsname{Table of contents}
\fi
\ifdefined\listfigurename
  \renewcommand*\listfigurename{List of Figures}
\else
  \newcommand\listfigurename{List of Figures}
\fi
\ifdefined\listtablename
  \renewcommand*\listtablename{List of Tables}
\else
  \newcommand\listtablename{List of Tables}
\fi
\ifdefined\figurename
  \renewcommand*\figurename{Figure}
\else
  \newcommand\figurename{Figure}
\fi
\ifdefined\tablename
  \renewcommand*\tablename{Table S}
\else
  \newcommand\tablename{Table}
\fi
}
\@ifpackageloaded{float}{}{\usepackage{float}}
\floatstyle{ruled}
\@ifundefined{c@chapter}{\newfloat{codelisting}{h}{lop}}{\newfloat{codelisting}{h}{lop}[chapter]}
\floatname{codelisting}{Listing}
\newcommand*\listoflistings{\listof{codelisting}{List of Listings}}
\makeatother
\makeatletter
\@ifpackageloaded{caption}{}{\usepackage{caption}}
\@ifpackageloaded{subcaption}{}{\usepackage{subcaption}}
\makeatother
\makeatletter
\@ifpackageloaded{tcolorbox}{}{\usepackage[skins,breakable]{tcolorbox}}
\makeatother
\makeatletter
\@ifundefined{shadecolor}{\definecolor{shadecolor}{rgb}{.97, .97, .97}}
\makeatother
\makeatletter
\makeatother
\makeatletter
\makeatother
\ifLuaTeX
  \usepackage{selnolig}  % disable illegal ligatures
\fi
\IfFileExists{bookmark.sty}{\usepackage{bookmark}}{\usepackage{hyperref}}
\IfFileExists{xurl.sty}{\usepackage{xurl}}{} % add URL line breaks if available
\urlstyle{same} % disable monospaced font for URLs
\hypersetup{
  pdftitle={Supplementary Material:},
  pdfauthor={John Waczak},
  colorlinks=true,
  linkcolor={blue},
  filecolor={Maroon},
  citecolor={Blue},
  urlcolor={Blue},
  pdfcreator={LaTeX via pandoc}}

\title{Supplementary Material:}
\usepackage{etoolbox}
\makeatletter
\providecommand{\subtitle}[1]{% add subtitle to \maketitle
  \apptocmd{\@title}{\par {\large #1 \par}}{}{}
}
\makeatother
\subtitle{Hyperspectral Reflectance Indices}
\author{John Waczak}
\date{2024-01-03}

\begin{document}
\maketitle
\ifdefined\Shaded\renewenvironment{Shaded}{\begin{tcolorbox}[boxrule=0pt, sharp corners, breakable, interior hidden, borderline west={3pt}{0pt}{shadecolor}, enhanced, frame hidden]}{\end{tcolorbox}}\fi

For the following hyperspectral reflectance indices, we have made the
following identifications:

\begin{align}\label{eq:ref-bands}
\begin{split}
    R_b &= R(440 \text{ nm}) \\
    R_g &= R(550 \text{ nm}) \\
    R_b &= R(650 \text{ nm}) \\
    R_{nir} &= R(860 \text{ nm}).
\end{split}
\end{align} We also define \(R_{far} = R(1009 \text{ nm})\) as the band
representing the farthest infrared bin for our hyperspectral imager
which we use instead of the usual SWIR band.

\begin{longtable}[]{@{}
  >{\raggedright\arraybackslash}p{(\columnwidth - 4\tabcolsep) * \real{0.3913}}
  >{\raggedright\arraybackslash}p{(\columnwidth - 4\tabcolsep) * \real{0.1304}}
  >{\raggedright\arraybackslash}p{(\columnwidth - 4\tabcolsep) * \real{0.4783}}@{}}
\caption{Spectral indices supplied as extra features to each ML model.
For each index, \(R_{\lambda}\) denotes the reflectance at wavelength
\(\lambda\) used to compute the index. \(R_b\), \(R_g\), etc are defined
in Equation \ref{eq:ref-bands}.}\tabularnewline
\toprule\noalign{}
\begin{minipage}[b]{\linewidth}\raggedright
\textbf{Spectral Index}
\end{minipage} & \begin{minipage}[b]{\linewidth}\raggedright
\textbf{Acronym}
\end{minipage} & \begin{minipage}[b]{\linewidth}\raggedright
\textbf{Formula}
\end{minipage} \\
\midrule\noalign{}
\endfirsthead
\toprule\noalign{}
\begin{minipage}[b]{\linewidth}\raggedright
\textbf{Spectral Index}
\end{minipage} & \begin{minipage}[b]{\linewidth}\raggedright
\textbf{Acronym}
\end{minipage} & \begin{minipage}[b]{\linewidth}\raggedright
\textbf{Formula}
\end{minipage} \\
\midrule\noalign{}
\endhead
\bottomrule\noalign{}
\endlastfoot
Difference Vegetation Index & DVI &
\(\dfrac{2.5(R_{nir} - R_r)}{R_{nir} + 6R_r - 7.5R_b + 1}\) \\
Global Environmental Monitoring Index & GEMI* &
\(\text{eta}(1 - 0.25\,\text{eta}) - \dfrac{R_r - 1.125}{1 - R_r}\) \\
Green Atmospherically Resistant Index & GARI** &
\(\dfrac{R_{nir} - (R_g - \gamma(R_b - R_r))}{R_{nir} + (R_g - \gamma (R_b - R_r))}\) \\
Green Chlorophyll Index & GCI & \(\dfrac{R_{nir}}{R_g} - 1\) \\
Green Difference Vegetation Index & GDVI & \(R_{nir} - R_g\) \\
Green Leaf Index & GLI &
\(\dfrac{(R_g - R_r) + (R_g - R_b)}{2 R_g + R_r + R_b}\) \\
Green Normalized Difference Vegetation Index & GNDVI &
\(\dfrac{R_{nir} - R_g}{R_{nir} + R_g}\) \\
Green Optimized Soil Adjusted Vegetation Index & GOSAVI &
\(\dfrac{R_{nir} - R_g}{R_{nir} + R_g + 0.16}\) \\
Green Ratio Vegetation Index & GRVI & \(\dfrac{R_{nir}}{R_g}\) \\
Green Soil Adjusted Vegetation Index & GSAVI &
\(\dfrac{1.5(R_{nir} - R_g)}{R_{nir} + R_g + 0.5}\) \\
Infrared Percentage Vegetation Index & IPVI &
\(\dfrac{R_{nir}}{R_{nir} + R_r}\) \\
Leaf Area Index & LAI &
\(3.618 \left(\dfrac{2.5 (R_{nir} - R_r)}{R_{nir} + 6R_{R_r} - 7.5 R_b + 1}\right) - 0.118\) \\
Modified Non-Linear Index & MNLI &
\(\dfrac{1.5(R_{nir}^2 - R_r)}{R_{nir}^2 + R_r + 0.5}\) \\
Modified Soil Adjusted Vegetation Index 2 & MSAVI2 &
\(\dfrac{2R_{nir} + 1 - \sqrt{(2R_{nir} + 1)^2 - 8(R_{nir} - R_r)}}{2}\) \\
Modified Simple Ratio & MSR &
\(\dfrac{R_{nir}/R_r - 1}{\sqrt{R_{nir} / R_r} + 1}\) \\
Non-Linear Index & NLI & \(\dfrac{R_{nir}^2 - R_r}{R_{nir}^2 + R_r}\) \\
Normalized Difference Vegetation Index & NDVI &
\(\dfrac{R_{nir} - R_r}{R_{nir} + R_r}\) \\
Normalized Pigment Chlorophyll Index & NPCI &
\(\dfrac{R_{680} - R_{430}}{R_{680} + R_{430}}\) \\
Optimized Soil Adjusted Vegetation Index & OSAVI &
\(\dfrac{R_{nir} - R_r}{R_{nir} + R_r + 0.16}\) \\
Renormalized Difference Vegetation Index & RDVI &
\(\dfrac{R_{nir} - R_r}{\sqrt{R_{nir} + R_r}}\) \\
Soil Adjusted Vegetation Index & SAVI &
\(\dfrac{1.5(R_{nir} - R_r)}{R_{nir} + R_r + 0.5}\) \\
Simple Ratio & SR & \(\dfrac{R_{nir}}{R_r}\) \\
Transformed Difference Vegetation Index & TDVI &
\(\dfrac{1.5R_{nir} - R_r}{\sqrt{R_{nir}^2 + R_r + 0.5}}\) \\
Triangular Greenness Index & TGI &
\(\dfrac{(\lambda_r-\lambda_b)(R_r-R_g) - (\lambda_r-\lambda_g)(R_r - R_b)}{2}\) \\
Visible Atmospherically Resistant Index & VARI &
\(\dfrac{R_g - R_r}{R_g + R_r - R_b}\) \\
Wide Dynamic Range Vegetation Index & WDRVI &
\(\dfrac{0.2 R_{nir} - R_r}{0.2 * R_{nir} + R_r}\) \\
Atmospherically Resistant Vegetation Index & ARVI &
\(\dfrac{R_{800} - (R_{800} - 1(R_{450} - R_{680}))}{R_{800} + (R_{680} - 1 (R_{450} - R_{680}))}\) \\
Modified Chlorophyll Absorption Ratio Index & MCARI &
\(((R_{700} - R_{670}) - 2(R_{700} - R_{550}))\dfrac{R_{700}}{R_{670}}\) \\
Modified Chlorophyll Absorption Ratio Index Improved & MCARI2 &
\(\dfrac{1.5( 2.5(R_{800} - R_{670}) - 1.3 (R_{800} - R_{550}))}{\sqrt{(2R_{800} + 1)^2 - (6R_{800} - 5 \sqrt{R_{670}}) - 0.5}}\) \\
Modified Red Edge Normalized Difference Vegetation Index & MRENDVI &
\(\dfrac{R_{750} - R_{705}}{R_{750} + R_{705} - 2R_{445}}\) \\
Modified Red Edge Simple Ratio & MRESR &
\(\dfrac{R_{750} - R_{445}}{R_{705} - R_{445}}\) \\
Modified Triangular Vegetation Index & MTVI &
\(1.2 (1.2 (R_{800} - R_{550}) - 2.5 (R_{670} - R_{550}))\) \\
Red Edge Normalized Difference Vegetation Index & RENDVI &
\(\dfrac{R_{750} - R_{705}}{R_{750} + R_{705}}\) \\
Transformed Chlorophyll Absorption Reflectance Index & TCARI &
\(3\left((R_{700} - R_{670}) - 0.2(R_{700} - R_{550})\dfrac{R_{700}}{R_{670}}\right)\) \\
Triangular Vegetation Index & TVI &
\(0.5(120 (R_{750} - R_{550}) - 200 (R_{670} - R_{550}))\) \\
Vogelmann Red Edge Index 1 & VREI1 & \(\dfrac{R_{740}}{R_{720}}\) \\
Vogelmann Red Edge Index 2 & VREI2 &
\(\dfrac{R_{734} - R_{747}}{R_{715} + R_{726}}\) \\
Vogelmann Red Edge Index 3 & VREI3 &
\(\dfrac{R_{734} - R_{747}}{R_{715} + R_{720}}\) \\
Photochemical Reflectance Index & PRI &
\(\dfrac{R_{531} - R_{570}}{R_{531} + R_{570}}\) \\
Structure Insensitive Pigment Index & SIPI &
\(\dfrac{R_{800} - R_{445}}{R_{800} + R_{680}}\) \\
Structure Independent Pigment Index & SIPI1 &
\(\dfrac{R_{445} - R_{800}}{R_{670} - R_{800}}\) \\
Plant Senescence Reflectance Index & PSRI &
\(\dfrac{R_{680} - R_{500}}{R_{750}}\) \\
Anthocyanin Reflectance Index 1 & ARI1 &
\(\dfrac{1}{R_{550}} - \dfrac{1}{R_{700}}\) \\
Anthocyanin Reflectance Index 2 & ARI2 &
\(\left(\dfrac{1}{R_{550}} - \dfrac{1}{R_{700}}\right)R_{800}\) \\
Carotenoid Reflectance Index 1 & CRI1 &
\(\dfrac{1}{R_{510}} - \dfrac{1}{R_{550}}\) \\
Carotenoid Reflectance Index 2 & CRI2 &
\(\dfrac{1}{R_{510}} - \dfrac{1}{R_{700}}\) \\
Normalized Difference Water Index 1 & NDWI1 &
\(\dfrac{R_g - R_{nir}}{R_g + R_{nir}}\) \\
Normalized Difference Water Index 2 & NDWI2 &
\(\dfrac{R_{nir} - R_{far}}{R_{nir} + R_{far}}\) \\
Modified Normalized Difference Water Index & MNDWI &
\(\dfrac{R_g - R_{far}}{R_g + R_{far}}\) \\
Water Band Index & WBI & \(\dfrac{970}{900}\) \\
Anthocyanin Content Index & ACI & \(\dfrac{R_g}{R_{nir}}\) \\
Chlorophyll Index Red Edge & CIre & \(\dfrac{R_{nir}}{R_{705}} - 1\) \\
Modified Anthocyanin Reflectance Index & MARI &
\(\left(\dfrac{1}{R_{550}} - \dfrac{1}{R_{700}} \right)R_{nir}\) \\
Moisture Stress Index & MSI & \(\dfrac{R_{far}}{R_{nir}}\) \\
MERIS Terrestrial Chlorophyll Index & MTCI &
\(\dfrac{R_{753.75} - R_{708.75}}{R_{708.75} - R_{681.25}}\) \\
Normalzied Difference Infrared Index & NDII &
\(\dfrac{R_{nir} - R_{far}}{R_{nir} + R_{far}}\) \\
Normalized Difference Red Edge & NDRE &
\(\dfrac{R_{790} - R_{720}}{R_{790} + R_{720}}\) \\
Red Green Ratio Index & RGRI & \(\dfrac{R_r}{R_g}\) \\
Red Edge Vegetation Stress Index & RVSI &
\(\dfrac{R_{714} + R_{752}}{2} - R_{733}\) \\
\end{longtable}



\end{document}
