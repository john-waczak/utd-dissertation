\chapter{Introduction}


The rapid pace of global change poses a significant and ever-present threat to human prosperity. To facilitate the development of remediation technologies and enable effective mitigation strategies, we must make \textit{data-driven decisions}. However, the limitations posed by the lack of highly available, highly resolved data coupled together with the computational difficulties posed by direct simulation of physics \textit{at scale} severely constrains our ability to make the low uncertainty predictions needed to meaningfully address these challenges. The goal of this applied physics dissertation is to expand the boundary of what is possible in realm of sensing by leveraging machine learning in a \textit{principled way} to produce actionable insights. To that end, the guiding question of this work can be summarized succinctly as: \textit{How can we best utilize the measurements we have together with the physics we know to estimate the quantities we care about?}


\subsection{Global Change}


\subsection{}

- first we should discuss global change and the necessity for improved sensing and simulation
- justify the use of machine learning in order to fill the gaps where theory provides the motivation between what we measure and what we want to predict but no easily-represented physical relationship exists.
- this is a good place to bring in the
- we can then discuss the advent of physics-informed machine learning, and better, scientific machine learning to fill in the gaps and not be Snake Oil Salesmen (TM). In particular the role of numerical methods has not been subsumed by machine learning. At best these techniques are complementary and any performance claims need to be carefully benchmarked so as to not oversell the abilities of machine learning.
- Many of these emerging methods are promising but have seen little application to noisy, real-world data. One can only examine the Lorentz equations so much before transitioning to actual atmospheric dynamics
- We can do a paragraph on the specific areas we are trying to address: (1) using machine learning to establish the nonlinear relationship between measured spectra and constituent concentrations in water, (2) data-based techniques for evaluating the uncertainty of single sensor time series with application to low cost sensing, (3) physics based models for time series dynamics of Air Quality data (i.e. there will never be enough sensors on a low cost station to explain variability -- we would need a car presence sensor, etc... but we can model most of the dynamics and treat the rest as intermittent forcing), (3) how can we fuse sensing and simulation in order to infer the abundance of chemicals below detectable limits (in particular with application to indoor air chemistry).




Much of the current machine learning landscape is dominated by highly abstract problems like image recognition, language prediction

In particular, this work focuses on three key areas

we should not throw the baby out with the bath water. Nor should we attempt to reinvent the wheel... we can use machine learning to \textit{fill in the gaps} where our current theory can not easily provide a direct link between the data we have and the target we wish to model but a relationship is easily justified on physical grounds. Further, we can use our knowledge of physical laws to impose strong constraints on the space of available models that may significantly improve the capability of machine learning models to extrapolate beyond the boundaries of their training data sets. 

make the point that for physical theories are typically only amenable to analytic techniques when sufficiently constrained to domains with high symmetry or where nonlinear interactions can be controlled so that linear effects dominate over bounded of higher order effects (re: perturbation theory). For common real world scenarios where edges are sharp, functions aren't smooth, and approximating assumptions can not be readily applied, we typically must resort to numerical methods which seek to solve the dynamical laws on computational domains with well posed boundary and initial conditions.

In the first proposed project, we will apply machine learning techniques to provide a sensor calibration where theory suggests


To that end, this work focuses on improving sensing capabilities in three key domains: water quality, outdoor air quality, and indoor air quality.





physical sensing and physics-informed machine learning. The world we inhabit is undergoing rapid and transformative changes, with pressing environmental challenges demanding innovative solutions. At the heart of this quest lies the critical need for a deeper understanding of our complex environmental systems, coupled with the ability to derive actionable insights from the wealth of data at our disposal. This dissertation represents my contribution to addressing these challenges by bridging the gap between the physical world and advanced data-driven methodologies. Through the fusion of physical sensing technologies and machine learning, my research endeavors to unlock profound insights into environmental phenomena, ultimately facilitating informed decisions and sustainable practices in an ever-changing world.



This goal is pursued by applying physics informed approaches together with a suite of sensing and computational technologies, implementing the reusable paradigm of software defined sensors, i.e. physical sensing elements wrapped in a software layer. This software layer can serve a variety of purposes such as calibration and the provision of enhanced or derived data products. It is part of a broader effort in the MINTS-AI laboratory at the University of Texas at Dallas. Where MINTS-AI is an acronym, Multi-Scale Multi-Use Integrated Intelligent Interactive Sensing in Service of Society for Actionable Insights.

Comprehensive environmental sensing is a timely and beneficial endeavor for a variety of reasons. The growing awareness of major environmental issues such as climate change, pollution, and habitat loss necessitates effective environmental monitoring and management. Comprehensive environmental sensing can provide real-time data on air and water quality, weather patterns, and other environmental factors, assisting in the identification and resolution of environmental issues. This assists in the development and implementation of policies and strategies aimed at reducing environmental impact and increasing sustainability. Given that, for instance, air quality can have significant effects on human health, this has particular societal value.

Comprehensive sensing of the environment can improve decision-making. The real-time and accurate data provided by environmental sensors can aid in informed decision-making regarding various aspects such as traffic management, industrial regulation, and crop planning. For instance, data on air quality can be used to inform decisions about reducing pollution levels, while data on weather patterns can help farmers to plan their crops and reduce water usage. Comprehensive sensing of the environment can be instrumental in emergency response. Real-time data on weather patterns, air quality, water levels and resources, and seismic activity can help emergency responders to prepare for and respond to natural disasters such as hurricanes, floods, and earthquakes. The quick and accurate information can enable effective and timely response, potentially saving lives and reducing the impact of the disaster.

Many advances in technology have enabled the creation of comprehensive sensing systems that can monitor and analyze data from various sensors and devices in real-time. In this thesis we use a range of technologies including autonomous robotic teams [@Dunbabin2012; @Rubenstein2014; @Chen2017], hyperspectral imaging [@Plaza2009; @Li2018; @Zhu2017], mesh networks utilizing the Internet of Things (IoT) [@Gubbi2013; @Atzori2010; @Al-Fuqaha2015], machine learning (ML) [@Goodfellow2016; @LeCun2015; @Jordan2015], edge computing, high-performance computing,  wearable sensors and modern high-performance dynamic programming languages such as Julia [@Bezanson2017] designed for numerical and scientific computing. These technologies have facilitated the collection and processing of large amounts of data from multiple sources, resulting in more accurate and comprehensive environmental monitoring.



Today, the most well known application of machine learning techniques


%% \section{Completed Work}

%% \subsection{An Autonomous Robotic Team}
%% \subsection{A Distributed Network of Low-Cost Air Quality Monitors}

%% \section{Proposed Work}

%% \subsection{Robot Team}
%% \subsection{Air Quality Network}
%% \subsection{HEART Chamber}
