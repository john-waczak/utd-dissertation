\chapter{Physical Context}


\section{Water Quality}

\subsection{Properties of Aqueous Solutions}
\subsection{Reflectance Spectroscopy}
\subsection{Solar Geometry}

An explanation of relevant solar angles as well as their determination (i.e. the code I ported to Julia from Matlab script Dr. Lary supplied). We should also comment on the importance of

Use David's figure from his book to illustrate the solar geometry.



\section{Air Quality}
\subsection{Pollution and Particulate Matter}
discuss sources, evolution, global trends, etc...

\subsection{Indoor Air Quality}

This statement serves as the foundation for The Clean Air in Buildings Challenge and this RFI. We completely agree with this premise, but we believe the ‘order of operations’ - among ventilation, filtration, and air cleaning - should be the primary focus of inquiry, and effective and safe air cleaning innovations should be a focus deserving full consideration. We can reconcile multiple, often competing goals by informing and empowering schools and commercial buildings to adopt a  Clean First mentality, which we will explore in this response, as well as examine the ‘art of the possible’ by taking a critical look at the $21^{st}$ century innovations we believe will be essential to the practical pursuit and achievement of The Clean Air in Buildings Challenge goals/objectives. We will demonstrate the necessary considerations of a Clean First approach as well as highlight its additional benefits in this response, which include but are not limited to:


\begin{itemize}
\item Reduced mechanical system energy consumption.
\item Reductions in healthcare-acquired infections (HAIs) that are statistically significant in healthcare settings.
\item Real-world testing/proof of lower microbiological burden (total and culturable bacteria, fungi, and their spores) in addition to improved ventilation/filtration alone.
\item Reduced school absenteeism.
\item Increased resiliency to future pandemics and communicable diseases.
\item Reduction in airborne communicable disease and allergen loads.
\end{itemize}



The real challenges posed by global environmental change, combined with rising utility costs as a result of inflation and geopolitical shocks to energy supply chains, have created an implicit dichotomy between climate and public health goals. As Prof. Bahnfleth has said,

\section{Photolysis}


\section{Physics of Chemical Reactions: Chemical Reaction Kinetics}

