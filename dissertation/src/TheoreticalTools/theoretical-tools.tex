\chapter{Theoretical Tools}



\section{Automatic Differentiation}

perhaps this can be moved to the Computational tools section instead....

Use Chris Rackauckas and Chris Olah blogs to start with chain rule and justify automatic differentiation. Then describe reverse mode via gradient tapes and the pullback operator.

\subsection{Forward Mode AD}
\subsubsection{Dual Numbers}

Describe dual numbers because they are cool (and by extension, forward mode AD)

\subsection{Reverse Mode AD}




% maybe these should go into a section for "time-series techniques" and just make them subsections instead?
\section{Embedding Theorems}
Discuss Taken's embedding theorem and other relevant information for the time-series work.

\section{Koopman Theory}
Give an overview of continuous and discrete Koopman operator theory.





\section{Bayesian Statistics}
Derive Baye's rule and other important concepts

\section{Maximum Likelihood Estimation}

\section{KL-Divergence}

\section{Mathematical Stuctures}
A generic overview of role of mathematical structures in physics, machine learning, and mathematical modeling.

\section{Uncertainty Propagation}
Linear v.s. Nonlinear Techniques, measurements.jl

\section{Kernelization Methods}


\section{Dynmical Systems}
can quote paper on origins of term \textit{phase space}
\subsection{Chaos???}


\section{Endmember Modeling of Reflectance Spectra}
We can discuss previous attempts at modeling measured reflectance spectra via linear combinations of endmember spectra. Similarly, we can do a nice discussion of methods for \textit{discovering} the endmembers from the combined spectral. We can also discuss the limitations of linearity assumptions and outline where that clearly fails, i.e. turbid solutions, etc.


\section{Eigen-stuff and the Singular Value Decomposition}
Begin with a general description of the utility of eigen-stuff type analysis. Then explain how the SVD is the natural extension of this to non-square systems (rectangular matrices). Further expand by illustrating the application to principal component analysis. Make sure to comment on the general utility of decomposing a function/vector/signal into a (infinite) linear combination of (stationary) modes with simple time evolution.

