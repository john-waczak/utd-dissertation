\documentclass[doublespacing]{utdthesis}



\usepackage{microtype}
\usepackage{amsmath,amssymb,amsthm}
\usepackage{graphicx}
\graphicspath{{./figures}}
\usepackage{url}

\usepackage{xcolor}

\usepackage[authoryear]{natbib}
\bibliographystyle{chicago}
\setlength{\bibsep}{12pt plus 1pt minus 1pt}
\let\cite=\citep
\usepackage{rotating}
\usepackage{ifpdf}
\ifpdf
\usepackage{hyperref}
\fi

\providecommand{\hyperref}[2][]{#2}

\newenvironment{exampleclasscode}
               {\parindent=1cm\vskip0pt plus2pt minus0pt\begin{verse}}
               {\end{verse}\vskip0pt plus2pt minus0pt}




%% my personal macros for convenience
\newcommand{\R}{\mathbb{R}}
\newcommand{\E}{\mathbb{E}}
\newcommand{\D}{\mathcal{D}}
\newcommand{\N}{\mathcal{N}}
%%% End of personal macro definitions.



%%% The following definitions MUST come before the document begins.
%
\author{John Waczak}
% \title{Physical Sensing and Physics-based Machine Learning \\ for Actionable Environmental Insights}
\title{Physical Sensing and Physics-based Machine Learning for Actionable Environmental Insights}
\thesistype{Dissertation}  % or "Thesis"
\degreefull{Doctor of Philosophy}
\degreeabbr{PhD}
\subject{Physics}
\graduationmonth{December}
\graduationyear{2024}
\prevdegrees{BS} % comma-separated list of PREVIOUS degrees

% List committee members in order.  Mark chairpersons with a "*":
\committeemember*{David J. Lary}
\committeemember{Christopher Simmons}
\committeemember{David Lumley}
\committeemember{Lindsay King}
\committeemember{Joseph Izen}


%%% Beginning of actual thesis document.

\begin{document}

\frontmatter

\signaturepage

\copyrightpage{2024}

\begin{dedication} % optional
  \textbf{\textcolor{red}{UPDATE REQUIRED}}
\end{dedication}

\maketitle

\begin{acks}{September 2024}
  \textbf{\textcolor{red}{UPDATE REQUIRED}}
\end{acks}

\begin{abstract}

\textcolor{red}{UPDATE REQUIRED}
%% The rapid pace of global change poses a significant and ever present threat to human well-being. To facilitate the development of remediation technologies and to enable effective mitigation strategies, we must make data-driven decisions. However, the limitations posed by the lack of highly available, highly resolved data coupled together with the computational difficulties posed by direct simulation of physics at scale severely constrains our ability to make the low uncertainty predictions needed to meaningfully address these challenges in real time. This dissertation presents novel machine learning strategies for combining physics knowledge with data driven methods in three key case studies. In the first, we demonstrate the ability for a coordinated robotic team to estimate the concentration of chemicals-of-concern \textit{in real time} by using machine learning to map reflectance spectra captured by an autonomous aerial drone directly to chemical concentrations with associated uncertainty estimates. In the second study, we present a novel technique for using temporal variograms to estimate the intrinsic uncertainty of low cost air quality sensors directly from their time series. Additionally, we implement two physics informed machine learning methods to model these collected time series enabling the identification of acute pollution events by modeling them as the result of external forcing. Finally, in the third study, we present the most comprehensive analysis of indoor air quality to date, which includes multi-component observations, a detailed chemical reaction mechanism (including ion chemistry), an extensive evaluation of indoor photolysis, and full chemical data assimilation (both 4D Var and a full Kalman filter) with detailed multi-component error analysis.

\end{abstract}

\tableofcontents
\listoffigures % required if you have any figures
\listoftables % required if you have any tables

\mainmatter


\chapter{Introduction}\label{ch:intro}


\textcolor{red}{somewhere we should add a note about the intended audience. In paritcular I want to be throrough and include all relevant details and derivations that I wish I had when I started.}



What we want:

1. Introductory paragraph(s) describing the goal of the dissertation

Current methods for water and air quality analysis are limited in two key ways:
1. Insufficient quantity of data at relevant spatial, spectral, and temporal
resolutions
2. Simulating full physical dynamics is often computationally infeasible to
provide actionable insights fast enough to enable human action/intervention.



- The rapid pace of global change increasingly endangers human health and well-being
and threatens environmental health and stability
- Continued advancements in low-cost, mobile sensing have dramatically increased
our ability to characterize the environment at spatial and temporal resolutions
relevant to human-scale interactions
- Additionally, a plethora of satellite missions have been recently launched (or
will soon be) which promise to provide petabytes of multi-spectral imagery
- However, the data provided by these systems are often limited to specific
physical quantities (e.g. reflectances) which are only indirectly related to water quality and air
quality parameters which we actually care about
- To make sense of these data, one can apply physical models which relate
measurements to parameters of interest
- However, these models are limited by the physics \textit{as we currently
  understand it} making it challenging to account for as yet undiscovered
relationships impacting the data 
- Additionally, simulating the dynamics of physical systems at scale demands 
significant computational resources limiting real-time applicability.
- For example, ECMWF meteorological forecasts are provided a (add spatial and
temporal resolution) despite the fact the many studies across the scientific
literature suggest relevant spatial scales between 0-1 km and temporal scales at
the order of 10 seconds for air quality.
- Machine learning offers a suite of data-driven methods to take advantage of
large datasets to extract meaningful relationships between 


% Amidst the rapid pace of global change, accurately assessing water and air quality is crucial for protecting the environment and safeguarding human health and well-being.

% Traditional monitoring methods are often insufficient to capture the complex and
% dynamic nature of these environmental systems, necessitating the development of
% advanced data collection and modeling techniques.


\section{Dissertation Goals}


The goal of this dissertation is to advance physical sensing in service of society by demonstrating the use of new techniques from physics-based machine learning on a variety of real world data sets. Towards this end, this work presents a collection of case studies utilizing both supervised and unsupervised machine learning techniques in a variety of contexts together with physical sensing to produce actionable environmental insights. The applications of this research apply widely across many scientific domains including remote sensing, time series analysis, air quality, chemical kinetics, and data assimilation.




\section{Summary of Original Contributions}

\subsection{Publications}
- Robot team supervised 1
- Robot team unsupervised GTM
- Robot team unsupervised GSM
- HAVOK models for PM outlier detection and short-term foractasting

\subsection{Datasets}
\subsection{Code}










%\section{Motivation}
\section{Water Quality}


\section{Air Quality}
Research into air quality began in
earnest in the mid-20th century driven by growing concerns over the impact of
pollution, industrial emissions, and urban smog on public health. In the US, the
Clean Air Act of 1963 accelerated research efforts by empowering the federal
government with authority to address air pollution standards. However, at the
time  


- Multiple components which contribute to overall air quality
- Gasses are a big one
- Describe all of the parameters that go into the Air Quality Index
- We focuse

- Prabuddha et al. combine remote sensing observations for aerosol optical depth
with meteorological data to predict ground level PM 2.5 using machine learning
together with ground based-sensors \cite{prabuddha-pm-satellite}
- Average US values exceeded 9 $\mu g/m^3$ standard 20\% of the time with the
eastern US and California exceeding the limit over 50\% of the time during the
sampling period from Jan. 2020 to June 2023.

\subsection{What is Particulate Matter}

\subsection{Environmental Impact}

\textcolor{red}{include a picture of fog/haze}

\subsection{Human Impact}




\section{Physical Sensing}

The successful application of machine learning methods demands comprehensive, carefully curated data sets. In order for our modeling efforts to be successful, it is critical that we capture the subtle nuances of the phenomena we wish to describe. In this chapter, we outline the various physical sensing approaches used throughout this dissertation, all of which are part of a broader effort in the MINTS-AI laboratory at the University of Texas at Dallas. MINTS-AI is an acronym for Multi-Scale Multi-Use Integrated Intelligent Interactive Sensing in Service of Society for Actionable Insights. An graphical overview of the MINTS-AI sensing paradigm is outlined in Figure \ref{fig:mints-ai}.

\begin{figure}[!hbt]
  \centering
  \includegraphics[width=0.8\textwidth]{introduction/MINTS-sentinels.jpg}
  \caption{The MINTS-AI context engine is a sensing paradigm composed of flexible sensing sentinels spanning from remote sensing data products to autonomous robots and ground survey vehicles.}
  \label{fig:mints-ai}
\end{figure}

Of the variety of sensing sentinels listed above, this dissertation is primarily concerned with three key applications. The first is a team of autonomous robotic vehicles which we refer to as the \textit{robotic team}. The second is a network of distributed streaming sentinels comprised of low-cost air quality sensors. The third describes a reference sensor chamber for air quality evaluation which we use to develop a \textit{simulation sentinel}.






\section{Machine Learning}

- The Role of Machine Learning in the Era of Big Data

Big data in the physical sciences
Comment on the annual data volumes produced by
\begin{itemize}
  \item LANDSAT
  \item Sentinel
  \item CERN
  \item James Webb
  \item SDO AIA
  \item Medical Imaging (MRI, CT scans, etc...)
\end{itemize}

What is machine learning

Use of machine learning in the physical sciences

\begin{itemize}
  \item Remote sensing (inter-instrument calibration, classification, object identification, change monitoring via the NDVI and similar indices, etc.)
  \item Protein Folding
  \item Drug discovery
  \item Surrogate modeling (i.e. for PDE solvers - now very popular at NVIDIA)
\end{itemize}


\subsection{Supervised and Unsupervised ML}


\subsection{Physics-based Machine Learning}

Successful physical theories are predictive, explainable, and quantifiable (i.e.
uncertainty can be characterized). Most machine learning tasks involve abstract
data for which this can be challenging. A physics-based machine learning
approach involves model selection, uncertainty quantification, physical
reasonableness, etc..


The general problem of ML is hard with when no structure is assumed on data (include example of pit bull or potato)

For situations involving data from physical sensors describing real systesm, physics tells us there are underlying rules which govern the dynamics of real systems

When applying ML we should therefore place a high value on physics-based models, i.e. machine learning models tailord for the specific physical system, rather than generic abstract algorithms. This is important in all stages of the ML pipeline from feature selection (Robot team supervised), dimensionality reduction (i.e. the latent space of the GTM and GSM), model selection, model evaluation, etc. In summary:

\begin{itemize}
  \item Prefer interpretable models (e.g. deicsion trees) to black boxes
  \item Prefer probabilistic models (e.g. GTM vs SOM) to deterministic models (all data has uncertainty and no model is perfect)
  \item Prefer models which incorporate prior knoweldge of the physical system such as dynamical laws, symmetries, natural constraints, etc. This is specifically known as \textbf{physics-informed} machine learning or \textbf{scientific machine learning}  (SciML).
\end{itemize}



\section{Actionable Insights}

Combining physical sensing with machine learning methods allows us to \textit{put science in action} in order to drive positve societal outcomes.


\section{Dissertation Overview}


\chapter{Physical Sensing for Environmental Quality Assessment}\label{ch:physical-sensing}


\section{Coordinated Robot Teams}

\section{Hyperspectral Imaging}

\section{Low Cost Air Quality Sensors}

We can put the details for laser-based optical particle counters here

\section{Reference Grade Sensors for Indoor Air Quality}

Add details about the instruments used for the chemcial data assimilation study.




\chapter{Characterizing Water Quality with Autonomous Robotic Teams and Supervised Learning}\label{ch:robot-team-supervised}


\section{Motivation}

\section{Supervised Learning}
\subsection{Decision Trees}
\subsection{Ensemble Learning and Random Forests}
\subsection{Cross Validation}
\subsection{Conformal Prediction}
Conformalizing Quantile Regression
Conformalizing Scalar Uncertainty Estimates
\subsection{Permutation Importance}

\section{Rapid Processing of Hyperspectral Imagery}
\subsection{Georectification}
\subsection{Reflectance Conversion}
\subsection{Implementation and Timing Results}

\section{Study Overview}

\section{Results}

\section{Discussion}


\chapter{Unsupervised Characterization of Water Composition with Generative Topographic Mapping}\label{ch:robot-team-gtm}

\textcolor{red}{
  Unmanned aerial vehicles equipped with hyperspectral imagers have emerged as
  an essential technology for the characterization of inland water bodies. The
  high spectral and spatial resolutions of these systems enable the retrieval
  of a plethora of optically active water quality parameters via band ratio
  algorithms and machine learning methods. However, fitting and validating
  these models requires access to sufficient quantities of in situ reference
  data which are time-consuming and expensive to obtain. In this study, we
  demonstrate how Generative Topographic Mapping (GTM), a probabilistic
  realization of the self-organizing map, can be used to visualize
  high-dimensional hyperspectral imagery and extract spectral signatures
  corresponding to unique endmembers present in the water.  Using data
  collected across a North Texas pond, we first apply  GTM to visualize the
  distribution of captured reflectance spectra, revealing the small-scale
  spatial variability of the water composition. Next, we demonstrate how the
  nodes of the fitted GTM can be interpreted as unique spectral endmembers.
  Using extracted endmembers together with the normalized spectral similarity
  score, we are able to efficiently map the abundance of nearshore algae, as
  well as the evolution of a rhodamine tracer dye used to simulate water
  contamination by a localized source.
}



\section{Motivation}

\section{Unsupervised Learning}

\section{Principal Component Analysis}

\section{Self Organizing Maps}

\section{Generative Topographic Mapping}

\section{Study Overview}

\section{Results}

\section{Discussion}

\chapter{Nonlinear Endmember Extraction and Spectral Unmixing with Generative Simplex Mapping}\label{ch:robot-team-gsm}


\textcolor{red}{
  In this paper, we introduce a new model for non-linear endmember extraction
  and spectral unmixing of hyperspectral imagery called Generative Simplex
  Mapping (GSM). The model represents endmember mixing using a latent space
  with points sampled within a $(n-1)$-simplex corresponding to the abundance
  of $n$ unique sources. Points in this latent space are non-linearly mapped
  to reflectance spectra via a flexible function combining linear and
  non-linear mixing. Due to the probabilistic formulation of the GSM, spectral
  variability is also estimated by a precision parameter describing the
  distribution of observed spectra. Model parameters are determined using a
  generalized expectation-maximization algorithm. In the event of purely
  linear mixing, non-linear contributions are naturally driven to zero. The
  GSM outperforms three varieties of non-negative matrix factorization for
  both endmember extraction accuracy and abundance estimation on a synthetic
  data set of linearly mixed spectra from the USGS spectral library. In a
  second experiment, the GSM is applied to real hyperspectral imagery captured
  over a pond in North Texas. The model is able to accurately identify
  spectral signatures corresponding to near-shore algae, water, and rhodmaine
  tracer dye introduced into the pond to simulate water contamination by a
  localized source. Abundance maps generated using the GSM accurately track
  evolution of the dye plume as it mixes into the surrounding water.
}



\section{Motivation}

\section{Spectral Mixing Models}
\subsection{Linear Mixing}
\subsection{Bilinear Mixing}
\subsection{Postlinear Polynomial Mixing}
\subsection{Popular Models}
NCA, PPI, MVF, Autoencoders, etc...

\section{Nonnegative Matrix Factorization}

\section{GSM}

\section{Study Overview}

\section{Results}

\section{Discussion}


\chapter{Distributed Sensor Networks for Air Quality Assessment}\label{ch:air-network}

\section{Measurement of Particulate Matter}

\subsection{Measurement Methods}

Describe federal method (gravimetric analysis) and optical particle counter
designs based on Mie theory.

\subsection{Problems of Scale}

Describe why we need dense sensor networks, specifically to address the relevant
spatio-temporal scales that are missed by averaging to suggested EPA standards.

Cite Prabuddha's paper and discuss value of dense sensor networks for providing
in-situ reference data which can be used to calibrate remote sensing data
products - highlight the application of PACE e.g. for black-carbon.


% \section{Low Cost Sensors for Real-Time PM Monitoring}

% \subsection{Sensor Design}

% include discussion of limitations, e.g. known problem of hygroscopic growth,
% which is why we incorporate a battery of sensors including sensors for
% meteorological parameters which can be used to post-process PM data for humidity correction.

% \subsection{Sensor Networking}


\section{A Low Cost Sensor Network For Air Quality Monitoring}

The highly expensive cost to acquire, calibrate, and maintain reference grade air quality monitors makes it challenging to assess the importance of spatial and temporal variability on local air quality. Since factors such as weather, terrain, traffic, and the distribution of other sources can all effect local air quality, the development of low-cost sensing solutions is vital to address risks of poor air quality on local communities. To address this gap, we have developed a hierarchy of low-cost air quality monitors which we have deployed throughout the Dallas Fort-Worth (DFW) metroplex. In this section, we describe the relevant sensor types as well as a robust data processing and visualization pipeline developed to enable open access to high quality data.

The sensor network is comprised of a combination of two types of nodes: \textit{Central Nodes} and \textit{LoRa Nodes}. The central nodes are designed to be deployed in locations with where dedicated power is available. Each contains a variety of sensors including Particulate Matter, VOCs, $\mathrm{CO_2}$, $\mathrm{NO_X}$, ionizing radiation, incident light intensity, sound levels, as well as meteorological variables including temperature, pressure, relative humidity, and dew point. The powered Central Nodes are equipped with a cellular modem to facilitate data transfer from the field.

Each Central Node supports a collection of ~10 LoRa nodes (named for the LoRa long rage wireless transmission protocol) which can be separated by distances up to ~5 km or more if line of sight is established. These smaller sensors are self powered using a combination of battery and solar cells, and each measures a similar array of relevant air quality metrics including particulate matter concentrations, gas concentrations, and meteorological parameters. Designs for the two node types are illustrated in Figure \ref{fig:mints-nodes}.
\begin{figure}[!hbt]
  \begin{subfigure}{.5\textwidth}
    \centering
    \includegraphics[width=.8\linewidth]{air-network/central-node.png}
    \caption{A 3d model of a Central Node}
  \end{subfigure}
  \begin{subfigure}{.5\textwidth}
    \centering
    \includegraphics[width=.8\linewidth]{air-network/lora-node.png}
    \caption{A 3d model of a LoRa Node}
  \end{subfigure}
  \caption{Two types of nodes from the MINTS Air Quality network.}
  \label{fig:mints-nodes}
\end{figure}

Using the LoRaWAN protocol, the Central and LoRa nodes form a low power, wide area network by which data packets from each LoRa node are transmitted in real time to the nearest Central Node. Using their inbuilt cellular connection, the central nodes are then able to pass sampled data to a data processing backend using an MQTT publish-subscribe model. As the primary goal of this network is to provide detailed, real-time air quality data, we have developed a public facing website, \url{http://SharedAirDFW.com}, to visualize the current and historic measurement at each sensor location together with 6-hour wind forecasts from NOAA and weather radar. Figure \ref{fig:sharedair-site} shows a sample screenshot of the site.
\begin{figure}[!hbt]
  \centering
  \includegraphics[width=0.85\columnwidth]{air-network/sharedairdfw-homepage.png}
  \caption{The interactive SharedAirDFW website show casing real time data streams from the MINTS sensor network.}
  \label{fig:sharedair-site}
\end{figure}




\section{Towards Real-Time Assessment and Data-Driven Decisions}


\subsection{The Data Pipeline}

Describe docker and containerization. Goal: simple, maintainable framework which
can easily scale as additional sensors are incorporated.

The entire pipeline should be easily reproducible to enable local development
and make it straight forward to transition to cloud-based computing via Amazon
EC2 and other cloud services.


In order to automate data collection, processing, analysis, and long-term storage, a containerized pipeline was developed as illustrated in Figure \ref{fig:dashboards}.
\begin{figure}[!hbt]
  \begin{subfigure}{.2\textwidth}
    \centering
    \includegraphics[width=0.85\columnwidth]{air-network/docker-elements.png}
    \caption{}
  \end{subfigure}
  \begin{subfigure}{.8\textwidth}
    \centering
    \includegraphics[width=.8\linewidth]{air-network/grafana-dashboard-1.png}
    \caption{}
  \end{subfigure}
  \caption{The backend infrastructure developed to support the MINTS sensor network.}
  \label{fig:dashboards}
\end{figure}
This pipeline is composed of 5 tools that utilize containerization for reliable development and deployment.
\begin{itemize}
\item \textbf{NodeRed}: This tool developed by IBM allows the creation of complex data processing pipelines by defining directed acyclic graphs (DAGs) composed of individual processing nodes. We utlize this tool to subscribe to each sensor's relevant MQTT topic and decode the data packets into the relevant sensor measurements. At the end of each sensor pipeline, the data are then injected into the InfluxDB time series database. A key advantage of NodeRed is that the wide array of pre-existing nodes enables a no-code solution that is easy to maintain.
\item \textbf{InfluxDB}: This is a open source time series database optimized for large cardinality datasets. By storing processed data in InfluxDB, we are able to provide easy queryable access to real time and historic data.
\item \textbf{Grafana}: This tool is an open source visualization platform for creating interactive dashboards. We utilize Grafana to create detailed, real-time displays for each sensor in the network that we can use to observe \textit{all} incoming measurements as well as identify pollution events and monitor sensor status.
\item \textbf{Quarto}: This tool allows the generation of automated analysis reports by using a literate programming paradigm built on top of Jupyter Notebooks. By using quarto we are able to automatically perform daily, weekly, monthly, and annual analyses for each sensor in the network and collect the results into formatted PDFs or a static website. We are developing analysis using quarto together with Julia, Python, and R to facilitate detailed report generation and provide actionable insights.
\item \textbf{Open Storage Network}: With help from Dr. Chris Simmons, we have developed a pipeline for long term data storage utilizing the Open Storage Network to provide open access to historic sensor data stored in the popular S3 format.
\end{itemize}
Individual sensor dashboards are made publicly available at \url{http://mdash.circ.utdallas.edu:3000}. Historic data are updated daily at \url{https://portal.osn.xsede.org/s3browser//?bucket_path=https://ncsa.osn.xsede.org/ees230012-bucket01}.



%\subsection{Live Dashboards}


\subsection{Automated Reports}

Highlight reproducibility as a key analysis criterion







\chapter{Modeling Local Particulate Matter Dynamics with Time Delay Embeddings}\label{ch:havok}
% \chapter{Time-delay Embedding Models for Particulate Matter - Outlier Detection and Forecasting}\label{ch:havok}

The low-cost sensor network detailed in Chapter~\ref{ch:air-network} collects a
continuous stream of air quality data for a plethora of locations with high
temporal resolution ranging between $0.1$ to $1.0$ Hz. The real-time
visualization dashboards developed for the network offer immediate insight into
local air conditions. However, the challenge remains to extract actionable
insights from the large amounts of historical data. For example, we are concerned
with answering simple questions such as: \textit{What is the typical pattern of
  PM variability at this location}, \textit{Are changes in PM gradual or due to
  identifiable events?}, and \textit{Can we use historical PM measurements to
  provide insight into future air quality trends?}.

In this chapter, we propose a physics-based model rooted in Koopman operator
theory to address these questions by accurately capturing local PM dynamics.
Specifically, we extend the \textit{Hankel Alternative View of Koopman} (HAVOK)
framework introduced by Brunton et al. to apply to time series measurements of
PM data. In this data-driven approach PM dynamics are described via a simple
linear system with aperiodic external forcing. The forcing function extracted
from the model provides a clear method to identify abrupt pollution events from
historical data. The model can be used to provide accurate short-term forecasts.
Importantly, HAVOK models require few parameters, can be fit efficiently using
established numerical linear algebra routines, and are small enough to be
readily deployed directly onto sensor hardware.


\section{Motivation}


- Pronounced seasonal variability observed for PM 2.5 in China in both urban and
rural areas around Beijing \cite{pm-patterns-china}
- Significant diurnal variation. In urban areas the distribution is bimodal with
a first peak around 7:00 - 8:00 am and an evening peak between 7:00 and 11:00 pm
corresponding to anthroprogenic activities, i.e. rush hour traffic and decrease
in boundary layer height throughout the afternoon \cite{pm-patterns-china}
- Rural diurnal pattern is unimodal with significant peak between 5 and 11 pm.

- Investigations of PM 2.5 trends in Augsburg, Germany revealed a similar
unimodal trend with peak values in the late afternoon \cite{pm-patterns-germany}

- A 2003 study in New York city anlazyed PM 2.5 measurements from 20 stations
across the city. They observed strong seasonal, weekday, and diurnal cycles with
peak concentrations between 7-9 am and minimum concentrations between 4 and 6
am. \cite{pm-patterns-nyc}
- The observed pattern suggests anthropogenic factures are primarily responsible
for observed diurnal cycle while meteorological conditions also have some influence.

- Iskandaryan et al. reviewed machine learning approaches for air quality
prediction \cite{skandaryan-2020}.
- They found that 66\% of studies considered an hourly rate
- NN and regression methods were most popular

- Prabuddha et al. combine remote sensing observations for aerosol optical depth
with meteorological data to predict ground level PM 2.5 using machine learning
together with ground based-sensors \cite{prabuddha-pm-satellite}
- Average US values exceeded 9 $\mu g/m^3$ standard 20\% of the time with the
eastern US and California exceeding the limit over 50\% of the time during the
sampling period from Jan 2020 to Jun 2023.





Particulate matter (PM) dynamics exhibit complex behavior influenced by a
variety of factors, including meteorological conditions, human activities, and
atmospheric processes.

PM concentrations typically follow well-established
diurnal patterns, with peaks often occurring during morning and evening rush
hours due to increased vehicular emissions, and lower concentrations observed
during the midday when atmospheric dispersion is more effective (Zhang et al.,
2012).

Seasonal variations are also common, with higher PM levels in winter due to
factors like heating emissions and temperature inversions that trap pollutants
close to the ground (Cheng et al., 2013).

Numerous models have been developed to analyze and forecast PM levels.

These range from statistical models, such as autoregressive integrated moving
average (ARIMA) models (Box et al., 1994), to more sophisticated machine
learning approaches like neural networks, which can capture nonlinear
relationships in the data (Zhang et al., 2020).

Physics-based models, including chemical transport models (CTMs), provide
detailed representations of atmospheric processes but require substantial
computational resources and detailed input data (Seinfeld \& Pandis, 2016).

More recently, hybrid approaches combining machine learning with physical models
have shown promise in improving forecast accuracy by leveraging the strengths of
both methodologies (Bi et al., 2021).


Physical models for particulate matter (PM) modeling and forecasting are
primarily based on the detailed representation of atmospheric processes that
govern the emission, transport, chemical transformation, and removal of PM from
the atmosphere. One of the most widely used types of physical models is the
Chemical Transport Model (CTM). These models simulate the movement of air masses
and the complex chemical reactions that occur in the atmosphere, providing a
comprehensive framework to predict PM concentrations at various spatial and
temporal scales. Examples of CTMs include models such as the Community
Multiscale Air Quality (CMAQ) model and the Weather Research and Forecasting
coupled with Chemistry (WRF-Chem) model (Byun \& Schere, 2006; Grell et al.,
2005).


CTMs rely on inputs such as emission inventories, meteorological data, and
boundary conditions to simulate the dispersion and transformation of PM. These
models are highly sophisticated and can incorporate secondary aerosol formation,
deposition processes (both dry and wet), and interactions with other pollutants
such as ozone and nitrogen oxides. However, despite their detailed nature, CTMs
can be computationally expensive and highly sensitive to uncertainties in input
data and model parameterizations (Kumar et al., 2015).


Another class of physical models used for PM forecasting is the Eulerian
grid-based models, which divide the atmosphere into a 3D grid. These models
simulate the transport and diffusion of PM across grid cells using numerical
methods to solve the advection-diffusion equations. Eulerian models like CMAQ
are known for their ability to simulate large-scale pollution events, but they
can be limited in resolving fine-scale local events, especially in urban
environments (Binkowski \& Roselle, 2003).


In addition to CTMs and Eulerian models, Lagrangian particle dispersion models
(LPDMs) are also employed for PM forecasting. LPDMs track the trajectories of
individual air parcels or particles, providing a detailed simulation of how
pollutants travel through the atmosphere. Models like the Hybrid Single-Particle
Lagrangian Integrated Trajectory (HYSPLIT) model are often used for forecasting
PM concentrations by simulating the long-range transport of pollutants and
identifying the sources of PM observed at specific locations (Stein et al.,
2015). These models are particularly useful for source apportionment and
tracking pollution plumes over long distances.

Despite their strengths, purely physical models have limitations in forecasting
the nonlinear and stochastic nature of PM concentrations, especially for
short-term predictions. To overcome these challenges, hybrid models that combine
physical models with data-driven approaches (such as machine learning) have
gained popularity in recent years (Bi et al., 2021). These hybrid models aim to
leverage the process-based understanding of physical models with the pattern
recognition capabilities of statistical or machine learning models, improving
forecast accuracy and computational efficiency.




- Known dynamics of particulate matter, e.g. the diurnal cycle

- Physical models for particulate matter dynamics, e.g. simulation and modeling
approaches


- Methods for PM forecasting


- Cite the prabuddha paper for remote sensing inference

- Eulerian models: simulate a volumes of air (in some kind of lattice)

- Combine relevant atmospheric chemistry and meteorological processes --> CMAQ

- Lagrangian models: simulate individual particle trajectories

- Combine Eulerian and Lagrangian models into hybrid system to model both
transport (advection) and diffusion of plumes --> HYSPLIT

- Time series methods?

- ARIMA, NN approachs, etc...

- Dynamic Mode Decomposition and Koopman operator theory for time series analysis

- The HAVOK method... NOTE that it is promising, but to the best of our current
knowledge, there has not been any application of HAVOK to \textit{real}, noisy data.


% Particulate matter (PM) dynamics exhibit complex behavior influenced by a variety of factors, including meteorological conditions, human activities, and atmospheric processes. PM concentrations typically follow well-established diurnal patterns, with peaks often occurring during morning and evening rush hours due to increased vehicular emissions, and lower concentrations observed during the midday when atmospheric dispersion is more effective (Zhang et al., 2012). Seasonal variations are also common, with higher PM levels in winter due to factors like heating emissions and temperature inversions that trap pollutants close to the ground (Cheng et al., 2013). Numerous models have been developed to analyze and forecast PM levels. These range from statistical models, such as autoregressive integrated moving average (ARIMA) models (Box et al., 1994), to more sophisticated machine learning approaches like neural networks, which can capture nonlinear relationships in the data (Zhang et al., 2020). Physics-based models, including chemical transport models (CTMs), provide detailed representations of atmospheric processes but require substantial computational resources and detailed input data (Seinfeld & Pandis, 2016). More recently, hybrid approaches combining machine learning with physical models have shown promise in improving forecast accuracy by leveraging the strengths of both methodologies (Bi et al., 2021). 

% ---



- Dynamics of particulate matter


- Time series models for PM (how good are forecasts)


- Value of a physics-based models, e.g. chemical transport

- Limitations of these models are due to poorly resolved meteorological
obsevations, for example, high resolution ECMWF analysis are at 0.1 x 0.1 km
grids with the lowest vertical model layer hundreds of meters above the ground.

- Many of these are being developed but have yet to see application to
realistic, noisy datasets




It should be noted that
despite the rapid pace of development in the fields of data-driven and
scientific machine learning, many recently developed techniques like the
Universal Differential Equations (UDEs), Hamiltonian Neural Networks, and others
have yet to see wide spread application on noisy, real-world datasets. Our
secondary goal for this chapter is therefore to demonstrate how, with some
slight modifications, these techniques can be applied to real-world problems.


In order to provide actionable insights we must be able to effectively model the
dynamics of our collected time series. In a perfect world, we would measure all
relevant physical quantities such that the time evolution of local air quality
at each sensor could be obtained by simulating the relevant micro-physics.
However, low cost sensor networks are not equipped with all the necessary
reference grade instruments needed to perform such simulations; accurate winds
speed and direction sensors alone can cost hundreds to thousands of dollars and
remote sensing data products are often unreliable at the ground level (i.e. in
the human head space). We therefore are motivated to develop techniques to model
our collected time series using only the data provided at a single node. There
are many approaches to this task in the statistics and machine learning
literature including statistical models like ARIMA and deep learning methods
like Recurrent Neural Networks \cite{intro-to-time-series-models,
  time-series-rnns}. While these methods can often lead to robust short term
predictions, they do not incorporate prior physics knowledge and therefore are
not primed to take advantage of underlying dynamical laws. Recently two
interesting physics-informed, data driven techniques have been developed for
just this type of scenario. The first we shall examine is the so-called
\textit{Hankel Alternative View Of Koopman} (HAVOK) framework which extends the
principle of dynamic mode decomposition to nonlinear systems
\cite{brunton-havok}. The second is an technique dubbed the \textit{Hamiltonian
  Neural Network} which extends the notion of a Neural Ordinary Differential
Equation to allow a neural network to learn coordinate transformations or the
original time series data which satisfy \textit{Hamiltons equations}
\cite{greydanus-hnn}.




Note the additional constraint that our model should be both physically
interpretable and small enough to be easily trained and deployed at scale. While
complicated DNN based approaches such as deep recurrent networks may provide an
ability to forecast, the size and training times involved for these approaches
are prohibitively high if we wish to train and deploy these models directly to
sensors in the network in an automated fashion. Also note the other HAVOK paper
which attempts to use HAVOK for predictions which showed that the HAVOK model
provides a superior one step prediction compared to other state of the art
models.


\section{Hankel Alternative View of Koopman}


- Koopman operator theory for dynamical systems \cite{brunton-koopman-theory}
- Original HAVOK paper \cite{brunton-havok-orig}.
- HAVOK paper w/ ML for forcing \cite{havok-ml}
- sHAVOK and connections to frenet-serret frame \cite{havok-diffgeo}


\subsection{Global Motivation: Koopman Operator Theory}

\subsection{Time-Delay Embeddings and Taken's Theorem}
\subsection{Hankel Alternative View of Koopman}
\subsection{Local Motivation: Frenet-Serret Frame}
\subsection{Structured HAVOK}
\subsection{Simple Example: Lorenz System}


\begin{figure}[h]
  \centering
  \includegraphics[width=\columnwidth]{havok/0-havok-lorenz/2__lorenz-orig.png}
  \caption{(\textbf{a}) Time series for the $x$, $y$, and $z$ components of the
    Lorenz system. (\textbf{b}) 3-dimensional visualization of the attractor
    formed from $x$, $y$, and $z$ components. We note that there are two clear
    lobes corresponding to the two attractors of the system.}
  \label{fig:lorenz-time-series-orig}
\end{figure}


\begin{figure}[h]
  \centering
  \includegraphics[width=0.5\columnwidth]{havok/0-havok-lorenz/1b_A-B-sHAVOK.pdf}
  \caption{Fitted operators $\mathbf{A}$ and $\mathbf{B}$ for the Lorenz system
    learned by the HAVOK model corresponding to the linear dynamics and
    intermittent forcing.}
  \label{fig:lorenz-A-B-heatmap}
\end{figure}


\begin{figure}[h]
  \centering
  \includegraphics[width=\columnwidth]{havok/0-havok-lorenz/3__havok-embedding.pdf}
  \caption{(\textbf{a}) Time series for first 3 components $v\_i$ of the HAVOK
    model. Original time series for each component are shown in blue. Time
    series predicted by the HAVOK model are overlaid in red. (\textbf(b)) The
    attractor formed by the first 3 embedding coordinates. By Taken's theorem,
    this attractor is diffeomorphic to the original attractor.}
  \label{fig:lorenz-havok-embedding}
\end{figure}



\begin{figure}[h]
  \centering
  \includegraphics[width=0.75\columnwidth]{havok/0-havok-lorenz/6__forcing-statistics.pdf}
  \caption{Proabability density function for forcing time series learned by the
    HAVOK model. The PDF is compared to a Gaussian fit for the same data. The
    wide tails of the PDF reflect that the forcing is intermittent.}
  \label{fig:lorenz-forcing-stats}
\end{figure}


\begin{figure}[h]
  \centering
  \includegraphics[width=\columnwidth]{havok/0-havok-lorenz/7__attractor-w-forcing.pdf}
  \caption{(\textbf{a}) The original time series $x(t)$ plotted with the squared
  forcing time series $v_r(t)$. Red colors indicate forcing above a chosen
  threshold which accurately identify lobe-switching events. (\textbf{b}) The
  Lorenz attractor colored using the same scheme.}
  \label{fig:lorenz-attractor-forcing}
\end{figure}



\begin{figure}[h]
  \centering
  \includegraphics[width=0.9\columnwidth]{havok/0-havok-lorenz/9__timeseries-reconstruction.pdf}
  \caption{The reconstructed time series for $x(t)$ using the learned HAVOK model.}
  \label{fig:lorenz-reconstruction}
\end{figure}




\section{Study Overview}

- Describe dataset, e.g. the specific central node, focus on PM 2.5, and the
collection period (find continuous data during no-rain period during the summer
of 2023 to evaluate HAVOK model independent of problems introduced by
hygroscopic growth)


\begin{figure}[h]
  \centering
  \includegraphics[width=0.8\columnwidth]{havok/1-havok-pm/1__timeseries-full-short.pdf}
  \caption{Time series of PM $2.5$ measurements from Central Hub 4 located in
    Planeo, Tx starting on 2023-08-04. This time series was selected during the
    long period of no precipitation during 2023 in order to limit the impact of
    hygroscopic growth on observed values. Note the occasional, thin vertical
    spikes corresponding to acute pollution events.}
  \label{fig:pm-timeseries-orig}
\end{figure}


\begin{figure}[h]
  \centering
  \includegraphics[width=0.7\columnwidth]{havok/1-havok-pm/1__pca-explained-variance.pdf}
  \caption{Explained variance of components from a principal component
    decomposition of the PM 2.5 time series sorted in decreasing order. A red
    line is superimposed indicating a $1\%$ explained variance. All components
    past the sixth contribute less than $1\%$ to the total explained variance. }
  \label{fig:pm-timeseries-pca}
\end{figure}


\section{Results}


\begin{figure}[h]
  \centering
  \includegraphics[width=0.8\columnwidth]{havok/1-havok-pm/1b__train-test-timeseries-short.pdf}
  \caption{Train-test split used for training the HAVOK model. To prevent data
    leaks, i.e. information from the future being incorporated into forecasts,
    the data was partitioned into two continuous sets to be used separately for
    model fitting and evaluation.}
  \label{fig:pm-train-test-split}
\end{figure}



\begin{table}[h]
  \caption{Results of HAVOK model hyperparameter search. Models were trained
    varying the embedding size ($N$), number of state variables $(r)$, and
    number of forcing terms ($n$). The top 10 models are presented here as
    evaluated by their RMSE and MAE values ranked in descending order.
    Additionally the top 3 models with $n=1$ are included for comparison.}
  \label{tab:havok-fit-results}
  \centering
  \begin{tabular}{ccccc} \hline
    \textbf{N} & \textbf{r} & \textbf{n} & \textbf{RMSE}  & \textbf{MAE} \\ \hline
    $30$ & $6$ & $5$ & $0.216703$ & $0.155912$ \\
    $45$ & $10$ & $5$ & $0.3649$ & $0.268842$ \\
    $30$ & $6$ & $4$ & $0.472797$ & $0.397746$ \\
    $45$ & $8$ & $5$ & $0.487448$ & $0.358938$ \\
    $30$ & $7$ & $5$ & $0.583654$ & $0.518485$ \\
    $30$ & $8$ & $5$ & $0.58551$ & $0.428986$ \\
    $30$ & $7$ & $4$ & $0.586575$ & $0.520145$ \\
    $30$ & $8$ & $4$ & $0.610296$ & $0.447673$ \\
    $30$ & $8$ & $3$ & $0.620928$ & $0.455369$ \\
    $60$ & $12$ & $5$ & $0.697797$ & $0.507499$ \\
    $\vdots$ & $\vdots$ & $\vdots$ & $\vdots$ & $\vdots$ \\
    $105$ & $3$ & $1$ & $1.48989$ & $1.09792$  \\
    $120$ & $3$ & $1$ &  $1.58473$ & $1.1494$ \\
    $180$ & $5$ & $1$ & $1.77659$ & $1.29506$ 
  \end{tabular}
\end{table}



\begin{figure}[h]
  \centering
  \includegraphics[width=0.75\columnwidth]{havok/2-havok-eval/1__predicted-ts-training.pdf}
  \caption{Time series predicted by HAVOK model with forcing functions provided.
  Inset into the figure is a subset of the }
  \label{fig:pm-havok-predictions}
\end{figure}




\begin{figure}[h]
  \centering
  \includegraphics[width=0.65\columnwidth]{havok/2-havok-eval/2__A-B-Havok.pdf}
  \caption{Operators learned by the HAVOK model with $r=6$ and $n=5$. We note
    that the $\mathbf{A}$ matrix displays the expected banded diagonal
    structure. Additionally, the values of the forcing matrix $\mathbf{B}$
    decrease with each column}
  \label{fig:pm-havok-operators}
\end{figure}



\begin{figure}[h]
  \centering
  \includegraphics[width=0.35\columnwidth]{havok/2-havok-eval/3__A-eigvals.pdf}
  \caption{Eigenvalues of the learned HAVOK operator $\mathbf{A}$ visualized in
    the complex plane. We note that no eigenvalues had any postive real
    component reflecting the model's stability for integration over long time periods. }
  \label{fig:pm-havok-eigvals}
\end{figure}



\begin{figure}[h]
  \centering
  \includegraphics[width=0.65\columnwidth]{havok/2-havok-eval/4__forcing-statistics.pdf}
  \caption{Probability density function evaluated for the first forcing function
  compared to a zero-mean Gaussian distribution fit to the forcing data. The
  wide tails of the estimated distribution reflect the intermittent activation
  of forcing.}
  \label{fig:pm-forcing-stats}
\end{figure}


\begin{figure}[h]
  \centering
  \includegraphics[width=\columnwidth]{havok/2-havok-eval/forcing-time-series.pdf}
  \caption{(\textbf{a}) Time series of first forcing function activation for the
    duration of the training set. (\textbf{b}) The same time series zoomed in to
    the first hours. The mean value for the forcing function was $7.94\times
    10^5$ reflecting the fact that the learned forcing functions tend to be
    centered about zero with no long-term trend.}
  \label{fig:pm-forcing-time-series}
\end{figure}


\begin{figure}[h]
  \centering
  \includegraphics[width=\columnwidth]{havok/2-havok-eval/6__timeseries-with-forcing.pdf}
  \caption{The original PM 2.5 time series plotted together with the squared
    value of the first forcing function, $\lVert  f_1(t) \rVert^2$. By
    thresholding the values of $f_1$, the intermittent PM spikes are easily
    identified. }
  \label{fig:pm-time-series-w-forcing}
\end{figure}




\begin{table}[h]
  \caption{Evaluation of HAVOK forecasting performance as a function of
    prediction horizon from 10 seconds to 4 minutes.}
  \label{tab:havok-forecasting-results}
  \centering
  \begin{tabular}{ccccccc} \hline
    \textbf{Duration} & \textbf{RMSE}  & \textbf{RMSE} & \textbf{MAE}   & \textbf{MAE}  & \textbf{MAPE}  & \textbf{MAPE} \\
                      & \textbf{train} & \textbf{test} & \textbf{train} & \textbf{test} & \textbf{train} & \textbf{test} \\\hline
    10 sec.	  & 0.0611 & 0.0589 & 0.0415 & 0.0415 & 0.0054 & 0.0057 \\
    1 min.	  & 0.2555 & 0.2398 & 0.1757 & 0.1720 & 0.0231 & 0.0235 \\
    2 min.	  & 0.7056 & 0.6764 & 0.4874 & 0.4945 & 0.0640 & 0.0674 \\
    3 min.	  & 1.0552 & 1.0366 & 0.7290 & 0.7691 & 0.0957 & 0.1052 \\
    4 min.	  & 1.3759 & 1.3567 & 0.9530 & 1.0185 & 0.1252 & 0.1389 \\
    5 min. 	  & 1.6696 & 1.6530 & 1.1611 & 1.2475 & 0.1525 & 0.1703
  \end{tabular}
\end{table}






\begin{figure}[h]
  \centering
  \includegraphics[width=\columnwidth]{havok/3-forcing-fit/forecast-scatter.pdf}
  \caption{Scatter diagrams for multistep forecasts. (\textbf{a}) 10 second
    forecast. (\textbf(b)) 1 minute forecast. (\textbf{c}) 2 minute forecast.
    (\textbf{d}) 5 minute forecast. The integration time step was $\Delta t =
    10$ s so that a 5 minute forecast corresponds to a 30 step future prediction.}
  \label{fig:pm-time-series-w-forcing}
\end{figure}


\chapter{A Chemical Data Assimilation Framework for Indoor Air Quality Assessment}\label{ch:autochem}


\section{Gaussian Processes}

\section{Adjoint Methods for Optimization}
\subsection{Linear Systems}
\subsection{Nonlinear Systems}
\subsection{Ordinary Differential Equations}


\section{Data Assimilation}
\subsection{Kalman Filter}
\subsection{Extended Kalman Filter}
\subsection{Continuous-discrete Extended Kalman Filter}
\subsection{3d-Var}
\subsection{4d-Var}

\section{Physics of Chemical Reactions: Chemical Reaction Kinetics}
\subsection{Overview}
\subsection{Chemical Equilibrium and the Law of Mass Action}
\subsection{Reaction Rate Laws}

\section{Summary of Chemical Mechanism Kinetics}

\section{Characterization of Photolysis Rates}
\subsection{Absorption Cross Sectionss}
\subsection{Quantum Yields}
\subsection{Photolysis Rate Determination}

\section{Chemical Data Assimilation}

\section{Results}

\section{Discussion}

\chapter{Future Work}\label{ch:future-work}

\section{Robot Team}

Minaturization of drones - next generation hyperspectral imagers (smaller) on smaller drones with additional payloads for visible and thermal imagers (not previously utilized)

Multi-sensor fusion and super resolution, e.g. use fine spatial resultion of visible imager together with spectral resolution of HSI and thermal to create a blended data product

Combine drone based imaging with remote sensing hyperspectral data, e.g. Enmap, PACE, etc...

Drone swarms for faster data collection

Additional data collection (can we test the generalizability across water bodies)

Additional ions with application to battery stuff...

\section{GTM}

GTM for guided data collection viat prize-collecting travelling salesman problem.

GTM for dust source identification. We can encourage finer cluserting behavior by augmenting the GTM to use adaptive mixing coefficients $\pi_k$ as we did for the GSM.

Batch version implementation of the GTM for \textit{big} datasets and online learning.

\section{GSM}

GSM for algal bloom identifiaction using remotely sensed hyperspectral imagery.

GSM for source apportionment of air quality measurements

Batch version of the GSM for \textit{big} dtaasets.


\section{PM Modeling}

Using HAVOK model to automatically identify pollution events and generate short term predictions near real time.

Exploring other physics informed models for analyzing PM data, e.g. Lagrangian
and Hamiltonian Neural Networks as another potential avenue -- Train on short
time scales and identify deviations of level surfaces of the Hamiltonian as a
means for identifying \textit{external forcing} like effects.

\section{Air Parcel Back Trajectories}

Compute back trajectories of PM data using meteorological analysis (and re-analyses).

Applications to human health outcomes: Alzheimers


\section{Chemical Data Assimilation}

Air quality chamber.






\chapter{Conclusions}\label{ch:conclusions}

\textcolor{red}{\textbf{Robot Team Supervised:}}
In this study, we address two key limitations of current remote sensing
approaches to characterize water quality: namely, the limited spatial, spectral,
and temporal resolution provided by existing satellite platforms and the lack of
comprehensive in situ measurements needed to validate remote sensing data
products. By equipping an autonomous USV with a suite of reference sensors, we
rapidly collect significantly more data than existing approaches that rely on
the collection of individual samples for lab analysis or are constrained to
continuous sensing at fixed sites. Utilizing an autonomous UAV equipped with a
hyperspectral imager in tandem with the USV allows us to quickly generate
aligned datasets that are used to train machine learning models mapping measured
reflectance spectra to the desired water quality variables. By virtue of this
increased data volume, we are able to simultaneously estimate the uncertainty of
our models by using conformal prediction. Finally, the hyperspectral data cube
processing workflow employed onboard the UAV makes it possible to deploy these
trained models to swiftly generate maps of the target variables across bodies of
water. The rapid turnaround time from data collection to model deployment is
critical for real-time water quality evaluation and risk assessment.




\textcolor{red}{\textbf{Robot Team GTM}}
In this study, we present  GTM as a useful unsupervised method for the visualization of UAV-based hyperspectral imagery and associated extraction of spectral endmembers. Using data collected at a North Texas pond, we demonstrate how the latent space of the GTM can be used to visualize the distribution of observed reflectance spectra revealing the small-scale spatial variability of water composition. Spectral signatures extracted from GTM nodes are used to successfully map the abundance of algae near the shore and to track the evolution of a rhodamine tracer dye plume. These examples illustrate the power of combining unsupervised learning with UAV-based hyperspectral imaging for the characterization of water composition. Future work will further develop the GTM as a tool to guide in situ data collection and enable contaminant localization for real-time applications.




\appendix % required only if you have appendixes
\chapter{Spectral Indices}\label{appendix:spectral-indices}

\chapter{Chemical Reaction Mechanisms}\label{appendix:mechanisms}

\textcolor{red}{UPDATE REQUIRED!}

%% \chapter{Reproducible Research Techniques}
\textcolor{red}{\textbf{UPDATE REQUIRED!}}

\section{Environment Management in Julia}
\section{Version Control (git)}
\section{CI/CD with github worfklows}
discuss automated tests as well as automatic document generation here

\section{Literate Programming and Automatic Documentation with Quarto}
\section{Containerization with Docker and Docker Compose}
Discuss NodeRed, InfluxDB, Grafana, etc...


%% \chapter{Optimization Methods}

Describe standard gradient descent and it's utility for machine learning. Expand to briefly describe the extensions used in our work:
\begin{itemize}
\item Gradient Descent
\item Gradient Descent with Momentum
\item ADAM
\item BFGS
\item LBFGS
\item The method we used for the variogram method that works specifically for quadratic loss functions...
\end{itemize}

We should also comment on which particular methods are best (and when)


%% \chapter{High Performance Computing}


Provide an overview of relevant concepts in high performance computing i.e.
\begin{itemize}
\item slurm
\item multi-threading
\item parallelization (distribured computing)
\item Memory management (i.e. preallocating data containers and writing functions that mutate, not allocate)
\end{itemize}


%% \chapter{Chemical Reaction Mechanisms}\label{appendix:mechanisms}

\textcolor{red}{UPDATE REQUIRED!}



% Begin the bibliography:
\begin{thesisbib}  % <--- THIS LINE IS REQUIRED!
  \bibliography{./references.bib}
\end{thesisbib}  % <-- THIS LINE IS REQUIRED!



\begin{biosketch}
  \textbf{\textcolor{red}{UPDATE REQUIRED!}}
\end{biosketch}


\begin{vita}  % <-- THIS LINE IS REQUIRED!
  \textbf{\textcolor{red}{UPDATE REQUIRED!}}
\end{vita}  % <-- THIS LINE IS REQUIRED!


\end{document}

